\documentclass[10pt, compress]{beamer}
\usetheme{m}
\usepackage{booktabs}
\usepackage[scale=2]{ccicons}
\usepackage{minted}
\usepackage{tabularx}
\usepackage{color}

\usepgfplotslibrary{dateplot}
\usemintedstyle{trac}

\title{Evaluación estratégica del uso de Gamification en ventas \emph{on-line}}
\subtitle{}
\date{27 de Abril, 2015}
\author{Ignacio J. Villacura de la Paz}
\institute{Departamento de Informática\\ Universidad Técnica Federico Santa María}

\begin{document}
\maketitle

\section{Introducción}
\begin{frame}[fragile]
    \frametitle{Contenidos}
    \begin{columns}[onlytextwidth]
        \column{0.5\textwidth}
            \begin{itemize}
                \item Motivación.
                \item \emph{Gamification}
                \begin{itemize}
                    \item Definición.
                    \item Usos cotidianos.
                \end{itemize}
                \item Problemática:
                \begin{itemize}
                    \item Definición.
                    \item Tienda ayudante.
                \end{itemize}
                \item Solución propuesta:
                \begin{itemize}
                    \item Descripción.
                    \item Sistema Base.
                    \item Herramientas utilizadas.
                    \item Encuesta evaluativa.
                \end{itemize}
            \end{itemize}
        \column{0.5\textwidth}
            \begin{itemize}
                \item Datos Experimentales:
                \begin{itemize}
                    \item Resultados implementación.
                    \item Resultados encuesta.
                \end{itemize}
                \item Conclusiones.
                \item Trabajos futuros.
            \end{itemize}
        \end{columns}
\end{frame}

\begin{frame}[fragile]
    \frametitle{Motivación}
    \begin{columns}[onlytextwidth]
        \column{0.5\textwidth}
            \begin{itemize}
                \item Entrada de multinacionales \\
                      al mercado nacional.
                \item PyMe.
                \item E-commerce y autogestión.
            \end{itemize}

        \column{0.5\textwidth}
            \begin{figure}
                \centering
                \includegraphics[width=1\textwidth]{images/pymes.jpg}
                \label{fig:awesom_image}
            \end{figure}
    \end{columns}
\end{frame}

\section{\emph{Gamification}}

\begin{frame}[fragile]
  \frametitle{Definición}
    \begin{center}
        \emph{``Es el uso de elementos del diseño de juego en contextos diferentes
        al de juego''}
    \end{center}

\begin{description}
 \item[\textbf{Juego}] Conjunto de reglas explicitas, que crean un ambiente de competición.
 \item[\textbf{Elementos del diseño de juego}] Se encuentran en la mayoría de juegos y cumplen un rol importante
en su jugabilidad.
 \item[\textbf{Contextos diferentes al de juego}] Existe fuera de un juego o de un ambiente que contiene \emph{gamification}.
\end{description}
\end{frame}

\begin{frame}
 \frametitle{Usos cotidianos}
\begin{center}
        \begin{itemize}
          \item Puntos \emph{retail}.
          \item Canje de productos.
	  \item Obtención de logros.
        \end{itemize}
\end{center}

\begin{columns}[onlytextwidth]
\column{0.5\textwidth}
\begin{figure}
\centering
    \includegraphics[width=0.8\textwidth]{images/retail.png}
    \caption{Puntos mas utilizados en Chile.}
    \label{fig:awesome_image}
\end{figure}

\column{0.5\textwidth}
\begin{figure}
\centering
    \includegraphics[width=0.8\textwidth]{images/gamAll.jpg}
    \caption{\emph{Gamification} en el mundo.}
    \label{fig:awesome_image}
\end{figure}

\end{columns}
\end{frame}

\section{Problemática}
\begin{frame}
\frametitle{Definición}
        \begin{itemize}
          \item Diferencias entre PyMes y grandes empresas, en ventas \emph{on-line}, debido a la diferencia de recursos.
	  \item PyMe y empleos.
	  \item Acceso de PyMe a tecnologías actuales para venta \emph{on-line}.
        \end{itemize}

\begin{table}[h]
\footnotesize
\centering
\begin{tabular}{|l|c|c|c|}
\hline
{\bf Empresa}  & {\bf \# de Trabajadores} & {\bf \% de ocupación} & {\bf Ventas anuales}\\
\hline
Micro    & 1 a 4                & 44.4\%  & menos de 2400 UF\\
\hline
Pequeña  & 5 a 49               & 37\%  & 2401 a 25000 UF\\
\hline
Mediana  & 50 a 199             & 13\%  & 25001 a 100000 UF\\
\hline
Grande   & más de 199           & 10\%  & más de 100000 UF\\
\hline
\end{tabular}
\caption{Tamaño de empresa según cantidad de empleados. (SOFOFA)}
\label{tab:tam_empresa}
\end{table}
\end{frame}

\begin{frame}
 \frametitle{Aplicación real}

Se contó con la ayuda de la tienda ``Kurgan''. Esta cedió el espacio para implementar el sistema de ventas
propuesto como solución.

\begin{columns}[onlytextwidth]
\column{0.5\textwidth}
\begin{figure}
\centering
    \includegraphics[width=0.8\textwidth]{images/logo.png}
    \caption{Logo tienda ``Kurgan''.}
    \label{fig:awesome_image}
\end{figure}

\column{0.5\textwidth}
\begin{figure}
\centering
    \includegraphics[width=1.0\textwidth]{images/mapa.png}
    \caption{Ubicación tienda ``Kurgan''}
    \label{fig:awesome_image}
\end{figure}

\end{columns}
\end{frame}

\section{Solución Propuesta}

\begin{frame}
 \frametitle{Descripción}

\begin{itemize}
 \item Sistema de ventas \emph{on-line}.
 \item Interfaz de administración simple.
 \item Utilización de \emph{gamification} para atraer y retener clientes.
\end{itemize}
\end{frame}

\begin{frame}
 \frametitle{Características Solución}
\begin{description}
 \item[\textbf{Bajo costo}] Debido a los escasos recursos que poseen las PyMes es necesario e importante que
la solución requiera los mínimos recursos posibles.
 \item[\textbf{Facilidad de configuración y administración}] El usuario no requiere mayores conocimientos al momento
de configurar y administrar el sistema.
\end{description}
\end{frame}

\begin{frame}
 \frametitle{Características Sistema Base}

\begin{itemize}
 \item Facilidad de instalación.
 \item Facilidad de administración.
 \item Actualizaciones y comunidad activa.
 \item Complementos Ad-hoc.
 \item Disponibilidad en varios idiomas.
\end{itemize}
\end{frame}

\begin{frame}
 \frametitle{Sistema Base}

\begin{table}[h]
\footnotesize
\setlength\extrarowheight{5pt}
\resizebox{\linewidth}{!}{
\begin{tabular}{| p{1.8cm} | p{1.6cm} | p{1.8cm} | p{1.8cm} | p{2.0cm} | p{1.8cm} |}
\hline
                        & Dificultad de\newline Instalación
                        & Dificultad de\newline Administración
                        & Actualizaciones\newline y Comunidad
                        & Complementos\newline Ad-hoc
                        & Disponibilidad\newline Idiomas\\ \hline

Magento                 & 7 & 8 & 8 & Si tiene plugins,\newline Alto costo & Parcialmente \vspace{0.2cm} \\ \hline
Prestashop              & 6 & 7 & 9 & No tiene plugins                     & Si    \vspace{0.2cm}\\ \hline
Wordpress +\newline Woocommerce & 4 & 5 & 2 & Si tiene plugins,\newline Gratuitas,\newline Bajo costo,\newline Alto Costo & Si\\ \hline
\end{tabular}}
\caption{Tabla de comparación entre los sistemas bases investigados}
\label{tab:comp_tools}
\end{table}

%\begin{figure}
%\centering
%    \includegraphics[width=0.9\textwidth]{images/tablaWord.png}
%    \caption{Tabla comparativa de sistemas bases.}
%    \label{fig:awesome_image}
%\end{figure}

\end{frame}

\begin{frame}
 \frametitle{Herramientas Utilizadas}

Las siguientes son las herramientas utilizadas para demostrar las ideas tras \emph{gamification}

\begin{itemize}
 \item \textbf{Woocommerce}: Herramienta que modifica el sistema base a una tienda on-line.
 \item \textbf{WooCube Pro}: Plugin dedicado al manejo de puntos.
 \item \textbf{WPAchievement}: Da la funcionalidad de entregar logros al realizar completamente una tarea.
 \item \textbf{Refer a Friend}: Herramienta que ayuda al usuario a invitar a amigos y obtener beneficios.
\end{itemize}
\end{frame}

\begin{frame}
 \frametitle{Encuesta Evaluativa}

Encuesta con los objetivos:

\begin{itemize}
 \item Apoyo de la investigación.
 \item Conocimiento de la población sobre \emph{gamification}.
 \item Interacción de la población con \emph{gamification}.
\end{itemize}
\end{frame}

\section{Resultados}

\begin{frame}
 \frametitle{Resultados Implementación}

En primera instancia, se debió establecer los beneficios a entregar:

\begin{itemize}
\item \textbf{Registro en la tienda:} 500 puntos.
\item \textbf{Puntos por venta:} 10\% del total de la compra.
\item \textbf{Comentarios:} 100 puntos.
\item \textbf{\emph{Referals}:} Cupón de 5\% en total venta.
\end{itemize}
\end{frame}

\begin{frame}
 \frametitle{Resultados Implementación}

\begin{block}{Pre-\emph{Gamification}}
Antes de implementar \emph{gamification} se obtuvieron datos por $30$ días para poder
compararlos una vez el sistema completo fuera implementado.
\end{block}

%\begin{itemize}
%    \item \textbf{Cantidad de visitas:} 315 (visitas únicas).
%    \item \textbf{Cantidad de lecturas:} 445.
%    \item \textbf{Cantidad de usuarios inscritos:} 0.
%    \item \textbf{Cantidad de ventas realizadas:} 0.
%\end{itemize}

\begin{block}{Post-\emph{Gamification}}
Una vez terminada la etapa de recolección de datos, se implemento \emph{gamification}. Esta
etapa tuvo una duración de $30$ días y se obtuvieron los siguientes datos:
\end{block}

%\begin{itemize}
%    \item \textbf{Cantidad de visitas:} 476 (visitas únicas).
%    \item \textbf{Cantidad de lecturas:} 641.
%    \item \textbf{Cantidad de usuarios inscritos:} 0.
%    \item \textbf{Cantidad de ventas realizadas:} 0.
%\end{itemize}

\end{frame}


\begin{frame}
 \frametitle{Resultados Implementación - Análisis de datos}

Al analizar los datos se puede concluir lo siguiente:

\begin{itemize}[<+- | alert@+>]
 \item Aumento en la cantidad de visitas.
 \item No hubo aumento de usuarios registrados.
 \item Implementación de \emph{gamification} \color{red}{no fue efectiva.}
\end{itemize}

\begin{table}[h]
\footnotesize
\centering

\resizebox{\linewidth}{!}{
\begin{tabular}{|l|c|c|c|c|}
\hline
  & {\bf \# visitas} & {\bf \# lecturas} & {\bf \# usuarios inscritos} & {\bf \# ventas realizadas}\\
\hline
Pre Implementación    & 315  & 445  & 0 & 0\\
\hline
Post Implementación  & 476 & 641  & 0 & 0\\
\hline
\end{tabular}}
\caption{Datos Pre y Post implementación.}
\label{tab:tam_empresa}
\end{table}

\end{frame}

\begin{frame}
 \frametitle{Resultados Encuesta}

Debido a que los datos no fueron concluyentes para decir si \emph{gamification} es una herramienta útil.
Se realizo una encuesta para complementar la información obtenida:

\begin{itemize}
    \item \textbf{Intervalo de confianza:} 95\%.
    \item \textbf{Error Muestral:} 7.5\%.
    \item \textbf{Cantidad mínima de respuestas:} 165\%.
    \item \textbf{Universo poblacional:} 180 personas.
    \item \textbf{Numero de Preguntas:} 12.
    \item \textbf{Sexo:} 71\% Masculino y 29\% Femenino.
\end{itemize}

\end{frame}

\begin{frame}
 \frametitle{Resultados Encuesta}

\begin{table}[h]
\centering
\footnotesize
\begin{tabular}{| p{6cm} | c | c | c |}
\hline
                          {\bf Pregunta}
                        & {\bf Si(\%)}
                        & {\bf No(\%)}
                        & {\bf Total} \\ \hline
¿Está al tanto de lo que es \emph{gamification}?&57\%&43\%&180 \\ \hline
¿Usted acumula puntos de grandes empresas?&56\%&44\%&180 \\ \hline
¿Está al tanto de los beneficios que dicha empresa le ofrece?&75\%&25\%&92 \\ \hline
¿Ha utilizado éstos beneficios en algún momento?&79\%&21\%&100 \\ \hline
¿Cree usted que el uso de \emph{gamification} lo motiva a volver a la tienda?&68\%&32\%&180 \\ \hline
\end{tabular}
\caption{Tabla de respuestas para preguntas de respuesta Si o No}
\end{table}
\end{frame}

\begin{frame}
 \frametitle{Resultados Encuesta}

\begin{table}[h]
\centering
\footnotesize
\resizebox{\linewidth}{!}{
\begin{tabular}{|p{5cm}|c|c|c|}
\hline
{\bf Pregunta} & {\bf Presencial } & {\bf On-line } & {\bf Total}\\
\hline
¿Cuál es su preferencia a la hora de realizar compras?& 26\% & 74\% & 180 \\
\hline
\end{tabular}}
\caption{Tabla de respuesta para pregunta de la encuesta: {\bf ¿Cuál es su preferencia a la hora de realizar compras?}}
\label{tab:Preg8}
\end{table}


\begin{table}[h]
\centering
\footnotesize
\resizebox{\linewidth}{!}{
\begin{tabular}{|p{4cm}|c|c|c|}
\hline
{\bf Pregunta} & {\bf Tienda on-line con \emph{gamification}} & {\bf Tienda on-line convencional} & {\bf Total}\\
\hline
¿Prefiere un sitio de ventas on-line con \emph{gamification} o un sitio convencional?& 69\% & 31\% & 180 \\
\hline
\end{tabular}}
\caption{Tabla de respuesta para pregunta de la encuesta: {\bf ¿Prefiere un sitio de ventas on-line con
\emph{gamification} o un sitio convencional?}}
\label{tab:Preg9}
\end{table}


\end{frame}

\begin{frame}
 \frametitle{Resultados Encuesta}

\begin{table}[h]
\centering
\footnotesize
\begin{tabular}{|l|c|c|c|c|c|c|}
\hline
 & \multicolumn{6}{c|}{{\bf Preferencias(\%)}} \\
\hline
{\bf Opciones} & {\bf 1} & {\bf 2} & {\bf 3} & {\bf 4} & {\bf 5} & {\bf Total}\\
\hline
Precios Bajos & 1\% & 1\% & 5\% & 13\% & 80\% & 180\\
\hline
Acumulación de puntos & 23\% & 22\% & 21\% & 17\% & 17\% & 180\\
\hline
Descuentos en compras & 7\% & 14\% & 18\% & 29\% & 32\% & 180\\
\hline
Obtención de productos exclusivos & 24\% & 17\% & 25\% & 16\% & 18\% & 180\\
\hline
Reconocimiento publico & 82\% & 7\% & 6\% & 2\% & 3\% & 180\\
\hline
\end{tabular}
\caption{Tabla de respuestas para pregunta de la encuesta: {\bf ¿Qué beneficios prefiere o preferiría obtener?}}
\label{tab:Preg7}
\end{table}

\end{frame}

\section{Conclusiones}

\begin{frame}
 \frametitle{Conclusiones}

\begin{itemize}
\item \emph{Gamification} es útil para las PyMes para poder atraer la atención de nuevos clientes..
\item Las herramientas utilizadas son adecuadas para una PyMe.
\item Es fundamental la participación y fomento del uso del sistema gamificado por el cliente y con esto
 entregar beneficios que al cliente le sean atractivos.
\item La población ya conoce el concepto y le atrae que las empresas lo utilicen.
\item Finalmente, los beneficios entregados regularmente son bastante atractivos pero la manera de entregarlos
debiese ser modificada.
\end{itemize}
\end{frame}

\begin{frame}
 \frametitle{Trabajos futuros}

\begin{itemize}

\item Realizar un \textbf{análisis de estrategias de marketing} que ayuden, en un inicio, a promocionar la nueva
implementación de \emph{gamification} en una tienda virtual.
\item Implementación de los otros sistemas bases con el objetivo de realizar una \textbf{comparación exhaustiva}
de estas herramientas.
\item Investigar y utilizar \textbf{otra forma de entrega de beneficios}.
\end{itemize}
\end{frame}

\maketitle

\end{document}
