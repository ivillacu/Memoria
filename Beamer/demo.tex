\documentclass[10pt, compress]{beamer}

\usetheme{m}

\usepackage{booktabs}
\usepackage[scale=2]{ccicons}
\usepackage{minted}
\usepackage{tabularx}

\usepgfplotslibrary{dateplot}

\usemintedstyle{trac}

\title{Evaluación estratégica del uso de Gamification en ventas \emph{on-line}}
\subtitle{}
\date{27 de Abril, 2015}
\author{Ignacio j. Villacura de la Paz}
\institute{Departamento de Informática\\ Universidad Técnica Federico Santa Maria}

\begin{document}

\maketitle

\section{Presentación actividad}

\begin{frame}[fragile]
  \frametitle{Actividad}
Durante la presentación se realizara una actividad con el fin de demostrar empiricamente 
el uso de \emph{gamification}.

Para esto necesitaran:

\begin{itemize}[<+- | alert@+>]
\item Papel.
\item Lapiz.
\item Atención a la presentación.
\end{itemize}

\end{frame}

\begin{frame}[fragile]
  \frametitle{Descripción actividad}

La actividad consiste en ir obteniendo puntos los cuales iran apareciendo durante la presentación.
Estos puntos los deben ir escribiendo en la hoja de seguimiento que se les entrego y al finalizar la presentación
se entregaran premios según el puntaje que obtengan.

\end{frame}

\begin{frame}[fragile]
  \frametitle{Ejemplo}
\begin{figure}

\textbf{Imagen del papel}

\centering
    \includegraphics[width=0.5\textwidth]{images/50-points.jpg}
    \caption{Awesome Image}
    \label{fig:awesome_image}
\end{figure}
\end{frame}


\section{Introducción}

\begin{frame}[fragile]
  \frametitle{Contenidos}

\begin{columns}[onlytextwidth]
 \column{0.5\textwidth}
  \begin{itemize}[<+- | alert@+>]
    \item Motivación.
    \item \emph{Gamification}
	\begin{itemize}[<+- | alert@+>]
	  \item Definición.
	  \item Usos cotidianos.
	\end{itemize}
    \item Problemática:
	\begin{itemize}[<+- | alert@+>]
          \item Definición.
          \item Tienda ayudante.
        \end{itemize}
    \item Solución propuesta:
	\begin{itemize}[<+- | alert@+>]
          \item Descripción.
          \item Sistema Base.
	  \item Herreamientas utilizadas.
	  \item Encuesta evaluativa.
        \end{itemize}
\end{itemize}

 \column{0.5\textwidth}
  \begin{itemize}[<+- | alert@+>]
    \item Datos Experimentales:
	\begin{itemize}[<+- | alert@+>]
          \item Resultados implementación.
          \item Resultados encuesta.
        \end{itemize}
    \item Conclusiones.
    \item Trabajos futuros.
  \end{itemize}
\end{columns}
\end{frame}

\begin{frame}[fragile]
  \frametitle{Motivación}
\begin{columns}[onlytextwidth]
 \column{0.5\textwidth}
\begin{itemize}
\item Entrada de multinacionales \\
al mercado nacional.
\item PyMe. 
\item E-commerce y autogestión.
\end{itemize}

 \column{0.5\textwidth}
\begin{figure}
\centering
    \includegraphics[width=0.8\textwidth]{images/SalesDashboard.jpg}
    \caption{Dashboard sistema de ventas.}
    \label{fig:awesome_image}
\end{figure}
\end{columns}
\end{frame}

\section{\emph{Gamification}}

\begin{frame}[fragile]
  \frametitle{Definición}
  \begin{description}
    \item[\textbf{\emph{Gamification}}] ``Es el uso de elementos del diseño de juego en contextos diferentes al de 
juego''.
  \end{description}

\end{frame}

\begin{frame}
\frametitle{Definición}

\begin{description}
 \item[\textbf{Juego}] En primer lugar este concepto se refiere al juego en su totalidad, y no a la acción de jugar. 
Este concepto es caracterizado por un conjunto de reglas explıcitas, que crean un ambiente en donde los 
jugadores buscan la competición para completar objetivos y metas.
 \item[\textbf{Elementos del diseño de juego}] Los elementos a utilizar son característicos a los juegos, que se 
encuentran en la mayoría de ellos y que cumplen un rol importante en su jugabilidad.
 \item[\textbf{Contextos diferentes al de juego}] Existe fuera de un juego o de un ambiente que contiene gamification.
\end{description}
\end{frame}

\begin{frame}
 \frametitle{Usos cotidianos}
\begin{columns}[onlytextwidth]
\column{0.5\textwidth}

        \begin{itemize}[<+- | alert@+>]
          \item Puntos \emph{retail}.
          \item Canje de productos.
	  \item Obtención de logros.
        \end{itemize}

\column{0.5\textwidth}
\begin{figure}
\centering
    \includegraphics[width=0.8\textwidth]{images/retail.png}
    \caption{Dashboard sistema de ventas.}
    \label{fig:awesome_image}
\end{figure}
\end{columns}
\end{frame}

\section{Problemática}
\begin{frame}
\frametitle{Definición}
        \begin{itemize}[<+- | alert@+>]
          \item Diferencias entre PyMes y grandes empresas, en ventas \emph{on-line}, debido a la diferencia de recursos.
	  \item PyMe y empleos.
	  \item Acceso de PyMe a tecnologias actuales para venta \emph{on-line}.
        \end{itemize}

\begin{table}[h]
\footnotesize
\centering
\begin{tabular}{|l|c|c|c|}
\hline
{\bf Empresa}  & {\bf Número de Trabajadores} & {\bf Porcentaje de ocupación} & {\bf Ventas anuales}\\
\hline
Micro    & 1 a 4                & 44.4\%  & menos de 2400 UF\\
\hline
Pequeña  & 5 a 49               & 37\%  & 2401 a 25000 UF\\
\hline
Mediana  & 50 a 199             & 13\%  & 25001 a 100000 UF\\
\hline
Grande   & más de 199           & 10\%  & más de 100000 UF\\
\hline
\end{tabular}
\caption{Tamaño de empresa según cantidad de empleados.}
\label{tab:tam_empresa}
\end{table}
\end{frame}

\begin{frame}
 \frametitle{Tienda}

Se conto con la ayuda de la tienda ``Kurgan''. Esta cedio el espacio para implementar el sistema de ventas 
propuesto como solución.

\begin{columns}[onlytextwidth]
\column{0.5\textwidth}
\begin{figure}
\centering
    \includegraphics[width=0.8\textwidth]{images/logo.png}
    \caption{Dashboard sistema de ventas.}
    \label{fig:awesome_image}
\end{figure}

\column{0.5\textwidth}
\begin{figure}
\centering
    \includegraphics[width=0.8\textwidth]{images/mapa.png}
    \caption{Dashboard sistema de ventas.}
    \label{fig:awesome_image}
\end{figure}

\end{columns}
\end{frame}

\section{Solución Propuesta}

\begin{frame}
 \frametitle{Descripción}

\begin{itemize}[<+- | alert@+>]
 \item Sistema de ventas online.
 \item Facil administracion.
 \item Utilización de \emph{gamification} para atraer y retener clientes.
\end{itemize}
\end{frame}

\begin{frame}
 \frametitle{Caracteristicas Solución}
\begin{description}
 \item[\textbf{Bajo costo}] Debido a los escasos recursos que poseen las PyMes es necesario e importante que
la solución tenga requiera los minimos recursos posibles.
 \item[\textbf{Facilidad de configuración}] El sistema debe poder ser configurado por el usuario.
 \item[\textbf{Facilidad de administración}] El sistema debe poder ser administrado por el usuario.
\end{description}
\end{frame}

\begin{frame}
 \frametitle{Caracteristicas Sistema Base}

\begin{itemize}[<+- | alert@+>]
 \item Facilidad de instalación.
 \item Facilidad de administración.
 \item Actualizaciones y comunidad activa.
 \item Características utilizables.
 \item Traducciones.
\end{itemize}
\end{frame}

\begin{frame}
 \frametitle{Sistema Base}

\begin{figure}
\centering
    \includegraphics[width=0.9\textwidth]{images/tablaWord.png}
    \caption{Dashboard sistema de ventas.}
    \label{fig:awesome_image}
\end{figure}

\end{frame}

\begin{frame}
 \frametitle{Herramientas Utilizadas}

\begin{itemize}
 \item Woocommerce.
 \item WooCube Pro.
 \item WPAchievement.
 \item Refer a Friend.
\end{itemize}

\begin{columns}[onlytextwidth]
\column{0.5\textwidth}
\begin{figure}
\centering
    \includegraphics[width=0.8\textwidth]{images/logo.png}
    \caption{Dashboard sistema de ventas.}
    \label{fig:awesome_image}
\end{figure}


\end{frame}
\begin{frame}[fragile]
  \frametitle{mtheme}

  The \emph{mtheme} is a Beamer theme with minimal visual noise inspired by the
  \href{https://github.com/hsrmbeamertheme/hsrmbeamertheme}{\textsc{hsrm} Beamer
  Theme} by Benjamin Weiss.

  Enable the theme by loading

  \begin{minted}[fontsize=\small]{latex}
    \documentclass{beamer}
    \usetheme{m}
  \end{minted}

  Note, that you have to have Mozilla's \emph{Fira Sans} font and XeTeX
  installed to enjoy this wonderful typography.
\end{frame}

\begin{frame}[fragile]
  \frametitle{Sections}
  Sections group slides of the same topic

  \begin{minted}[fontsize=\small]{latex}
    \section{Elements}
  \end{minted}

  for which the \emph{mtheme} provides a nice progress indicator \ldots
\end{frame}

\section{Elements}

\begin{frame}[fragile]
  \frametitle{Typography}
      \begin{minted}[fontsize=\small]{latex}
The theme provides sensible defaults to \emph{emphasize}
text, \alert{accent} parts or show \textbf{bold} results.
      \end{minted}

  \begin{center}becomes\end{center}

  The theme provides sensible defaults to \emph{emphasize} text,
  \alert{accent} parts or show \textbf{bold} results.
\end{frame}
\begin{frame}{Lists}
  \begin{columns}[onlytextwidth]
    \column{0.5\textwidth}
      Items
      \begin{itemize}
        \item Milk \item Eggs \item Potatos
      \end{itemize}

    \column{0.5\textwidth}
      Enumerations
      \begin{enumerate}
        \item First, \item Second and \item Last.
      \end{enumerate}
  \end{columns}
\end{frame}
\begin{frame}{Descriptions}
  \begin{description}
    \item[PowerPoint] Meeh.
    \item[Beamer] Yeeeha.
  \end{description}
\end{frame}
\begin{frame}{Animation}
  \begin{itemize}[<+- | alert@+>]
    \item \alert<4>{This is\only<4>{ really} important}
    \item Now this
    \item And now this
  \end{itemize}
\end{frame}
\begin{frame}{Figures}
  \begin{figure}
    \newcounter{density}
    \setcounter{density}{20}
    \begin{tikzpicture}
      \def\couleur{mLightBrown}
      \path[coordinate] (0,0)  coordinate(A)
                  ++( 90:5cm) coordinate(B)
                  ++(0:5cm) coordinate(C)
                  ++(-90:5cm) coordinate(D);
      \draw[fill=\couleur!\thedensity] (A) -- (B) -- (C) --(D) -- cycle;
      \foreach \x in {1,...,40}{%
          \pgfmathsetcounter{density}{\thedensity+20}
          \setcounter{density}{\thedensity}
          \path[coordinate] coordinate(X) at (A){};
          \path[coordinate] (A) -- (B) coordinate[pos=.10](A)
                              -- (C) coordinate[pos=.10](B)
                              -- (D) coordinate[pos=.10](C)
                              -- (X) coordinate[pos=.10](D);
          \draw[fill=\couleur!\thedensity] (A)--(B)--(C)-- (D) -- cycle;
      }
    \end{tikzpicture}
    \caption{Rotated square from
    \href{http://www.texample.net/tikz/examples/rotated-polygons/}{texample.net}.}
  \end{figure}
\end{frame}
\begin{frame}{Tables}
  \begin{table}
    \caption{Largest cities in the world (source: Wikipedia)}
    \begin{tabular}{lr}
      \toprule
      City & Population\\
      \midrule
      Mexico City & 20,116,842\\
      Shanghai & 19,210,000\\
      Peking & 15,796,450\\
      Istanbul & 14,160,467\\
      \bottomrule
    \end{tabular}
  \end{table}
\end{frame}
\begin{frame}{Blocks}

  \begin{block}{This is a block title}
    This is soothing.
  \end{block}

\end{frame}
\begin{frame}{Math}
  \begin{equation*}
    e = \lim_{n\to \infty} \left(1 + \frac{1}{n}\right)^n
  \end{equation*}
\end{frame}
\begin{frame}{Line plots}
  \begin{figure}
    \begin{tikzpicture}
      \begin{axis}[
        mlineplot,
        width=0.9\textwidth,
        height=6cm,
      ]

        \addplot {sin(deg(x))};
        \addplot+[samples=100] {sin(deg(2*x))};

      \end{axis}
    \end{tikzpicture}
  \end{figure}
\end{frame}
\begin{frame}{Bar charts}
  \begin{figure}
    \begin{tikzpicture}
      \begin{axis}[
        mbarplot,
        xlabel={Foo},
        ylabel={Bar},
        width=0.9\textwidth,
        height=6cm,
      ]

      \addplot plot coordinates {(1, 20) (2, 25) (3, 22.4) (4, 12.4)};
      \addplot plot coordinates {(1, 18) (2, 24) (3, 23.5) (4, 13.2)};
      \addplot plot coordinates {(1, 10) (2, 19) (3, 25) (4, 15.2)};

      \legend{lorem, ipsum, dolor}

      \end{axis}
    \end{tikzpicture}
  \end{figure}
\end{frame}
\begin{frame}{Quotes}
  \begin{quote}
    Veni, Vidi, Vici
  \end{quote}
\end{frame}


\section{Conclusion}

\begin{frame}{Summary}

  Get the source of this theme and the demo presentation from

  \begin{center}\url{github.com/matze/mtheme}\end{center}

  The theme \emph{itself} is licensed under a
  \href{http://creativecommons.org/licenses/by-sa/4.0/}{Creative Commons
  Attribution-ShareAlike 4.0 International License}.

  \begin{center}\ccbysa\end{center}

\end{frame}

\plain{Questions?}

\end{document}
