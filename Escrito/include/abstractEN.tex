The problem betewen small and big bussiness is really interesting. This is because the small ones are the ones
that provide a big percentage of the jobs in Chile. Since the big companies got into the economy of the country,
the small bussiness have been losing position on the market and with this the unemployments has grown.
This investigation focuses on the direfences that are between this two types of bussiness, small and big, 
in the matter of on-line sales. This is a sector where both sizes of companies can compete in the same terms but
a small bussiness needs a big amount of resources that does not have, from capital to knowledge.
To shorten this diference in this sector we will use {\gam} as a tool to attract and hold the customers
so they keep buying on the on-line shop.
This will be done by implementing a web system that groups an on-line shop and modules that interacts with
the customers displaying the ideas of {\gam}. In addition, as a support to the implementation of the system, 
was develop a survey with the objective to see how much was the knowledge of {\gam}.
After the implementation of the on-line shop, this had a poor reception with the people because of external
reason like the lack of interest from the owner of the bussiness and economic problems. From the survey,
it was obtained that the idea of {\gam} is very interesting for the population because they can get benefits
and is this the motivation they get to stay and keep buying on the on-line shop. Another important information
obtain on the survey is that is very necessary to search or develop a new way to give the benefits because
the accumulation of points is not very interest for the customer. 
Finally, the use of {\gam} can be highlighedt because it's help de bussiness to improve their sales because
the customer is very interested on getting benefits while they buy on the store. With this, small bussiness 
can compete with bigger ones.

Keywords: \textbf{\emph{Gamification}, \emph{E-commerce}, \emph{Wordpress}, \emph{Web Systems}}
