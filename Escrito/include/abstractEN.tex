The problem that small businesses have against larger companies is interesting. This is because Pymes  provide the largest amount of employment in Chile. Since multinational business entered the country, small business has been losing place so this has improved  the  rate of unemployment. This thesis  work focuses on the difference between these companies in terms of sales online. This is a sector which can compete on equal terms but a lot of resources are needed which  a small business does not have, from  knowledge to  money. To help lessen this gap in the field of online sales it will be used{\gam} as a tool to attract new customers and retain them with  the objectives to  incentives them to  buy  in the virtual store. That will be implementing a web system that provides the virtual store and modules that interact with the customer and thus show the ideas presented to us by {\gam}. 
In addition, to support the implementation of the system a survey was  developped  which aims are to see  the awareness of the population about {\gam}, the daily interaction with the concept of {\gam} and get the benefits that more  attract the  customers. 
After the implementation of the virtual store, this had a poor reception from the customers  due to external  factors  to the system such as lack of interest from the business Company  and economic problems. 
From the applied survey  was obtained that  people is  interested in the use  of {\gam} by businesses because these deliver benefits and this is what attracts and retains customers. 
An important point is necessary to find a new way of delivering benefits due to the accumulation of points is no longer a system interesting to the public. Finally, it can be noted that the use of {\gam} helps companies to improve their sales due to the interest of the client on  getting a  profit when buying in a store. With this small business  can more directly compete with larger companies.


Keywords: \textbf{\emph{Gamification}, \emph{E-commerce}, \emph{Wordpress}, \emph{Web Systems}}
