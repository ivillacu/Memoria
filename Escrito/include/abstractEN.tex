The problem betewen small and big bussiness is really interesting. This is because the small ones are the ones
that provide a big percentage of the jobs in Chile. Since the big companies got into the economy of the country,
the small bussiness have been losing position on the market and with this the unemployments has grown.
This investigation focuses on the direfences that are between this two types of bussiness, small and big, 
in the matter of on-line sales. This is a sector where both sizes of companies can compete in the same terms but
a small bussiness needs a big amount of resources that does not have, from capital to knowledge.




El problema que poseen las pequeñas empresas contra las empresas de mayor tamaño es interesante. Esto es debido
a que son las PyMe las que entregan la mayor parte del empleo en Chile. Desde que las multinacionales ingresaron
al país las pequeña empresa ha ido perdiendo terreno y desapareciendo y con ello a llevado a una mayor tasa de
desempleo.
Este trabajo de memoria se enfoca en la diferencia que existe entre estas empresas en lo que a ventas online
se refiere. Este es un sector en el cual se puede competir de igual a igual pero se necesita una gran cantidad de
recursos que una pequeña empresa no posee, desde capital a conocimiento.
Para ayudar a acortar esta diferencia en el sector de ventas on-line se utilizara {\gam} como herramienta para
atraer a nuevos clientes y retenerlos con el objetivos de que se incentiven en volver a comprar en la tienda virtual.
Eso se hará implementando un sistema web que contempla la tienda virtual y módulos que interactúan con el cliente
y así mostrar las ideas que nos presenta {\gam}. Además, para apoyar la implementación del sistema se desarrollo
una encuesta que tiene como objetivo ver el conocimiento de la población sobre {\gam}, la interacción diaria con el
concepto de {\gam} y obtener los beneficios que mas atraen a los clientes.
Luego de la implementación de la tienda virtual, esta tuvo una pobre recepción del publico debido a factores
externos al sistema como son la falta de interés por parte de la empresa y problemas económicos. De la encuesta
realizada se obtuvo que a la población le interesa el uso {\gam} por parte de las empresas debido a que estos
entregan beneficios y es esto lo que atrae y retiene a los clientes. Un punto importante es lo necesario
de buscar una nueva forma de entrega de beneficios debido a que la acumulación de puntos ya no es un sistema
interesante para el publico.
Finalmente, se puede destacar que el uso de {\gam} ayuda a las empresas a mejorar sus ventas debido a que al
cliente le interesa una forma de obtener beneficios al momento de comprar en alguna tienda. Con esto se puede
competir mas directamente con empresas de mayor tamaño.



Keywords: \textbf{\emph{Gamification}, \emph{E-commerce}, \emph{Wordpress}, \emph{Web Systems}}
