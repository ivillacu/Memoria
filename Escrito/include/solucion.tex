En esta secciòn se describira a fondo la solucion propuesta para mejorar la problemtica de las pequeñas
empresas para competir en lo referente a comercio electronico con sus pares mayores.
 \red{volver para escribir las secciones}

\section{Metodologia de la soluciòn}

La idea para dar solucion a la problematica de las micro empresas, para atraer y retener clientes y
con esto competir, con empresas de mayor tamaño tiene como base la utilizacion de herramientas gratis
o de bajo costo para crear una tienda virtual que logre competir con sitios web de empresas de mayor
tamaño. Debido a la bajo presupuesto que poseen los empresarios pequeños esta web debe cumplir con 
ciertas caracteristicas:

\begin{itemize}

\item Bajo costo: Debido a los pocos capitales que posee la micro o pequeña empresa, esta caracteristica es 
bastante importante. No es viable utilizar soluciones web con costos elevados que luego habra que desenvolsar
mas dinero para poder adecuarla a los requerimientos del empleador. 
\item Facilidad de configuracion: Existen diversas herramientas que ayudan a un negocio emergente a crear
una web, pero pocas estan pensadas para ser configuradas por el usuario. Esta caracteristica es importante
ya que es posible que la persona encargada de la instalacion y configuracion no sea conocedor de la tecnologia
y deba instruirse sobre la herramienta.
\item Facilidad de administracion: Caracteristica clave dentro de la solucion planteada. El sistema sera 
administrado, la mayoria del tiempo, por el dueño de la empresa que debe ser capaz de hacer tareas en la web
de manera rapida y si no posee el conocimiento debe ser facil de aprender y ejecutar. Esto debido a que el capital
humano en este tamaño de empresas, micro o pequeña, es disminuido y por lo tanto el tiempo dedicado a cada tarea
es menor.

\end{itemize}

Una vez creada la tienda virtual, la empresa entra a competir electronicamente, pero en desventaja,
con empresas de mayor capital y por ende con tiendas virtuales de mayor tamaño y que utilizan conceptos
para atraer y retener a los clientes. Para poder competir directamente, es necesario utilizar estos mismo 
conceptos para poder alcanzar, de cierta forma, el nivel de visitas y ventas de estos grandes sitios web.

Para poder disminuir la brecha entre la solucion propuesta y una tienda virtaul de gran tamaño utilizaremos el
concepto de \emph{gamification}. Esta idea ayudara a la tienda a atraer y retener clientes con la utilizacion
de conceptos del desarrollo de juegos. Los que se implementaran en la solucion seran:

\begin{itemize}

\item Puntos: Esta herramienta conciste en la entrega de puntos por la realizacion de tareas definidas o compras
 en la tienda virtual. Estos puntos son equivalentes a credito en la tienda y pueden ser utilizados como 
parte de pago. Estos son utilizados para que los clientes vuelvan a comprar a la tienda.
\item Achievements: Medallas coleccionables entregadas a los clientes realizando tareas definidas. Esta 
recompensa es utilizada con el fin de entregar sensacion de exito al cliente y asi tener motivacion para volver
a la tienda.
\item Referals: Invitaciones de usuarios de la tienda que son entregadas a posibles clientes para que estos 
conozcan y compren en la tienda. Esta herramienta entrega un beneficio mutuo tanto al que hizo la imvitacion
como al invitado una vez que realize una tarea definida. Esto es utilizado para atraer clientes con similares
gustos a la tienda.

\end{itemize}

Al unir la tienda virtual con las ideas propuestas dadas por \emph{gamification} se crea un sistema de ventas online
capaz de competir con tiendas que poseen mas capital y experiencia en este ambito.


\section{Herramientas}

Para implementar la solucion propuesta se necesito el uso de variadas herramientas. En un principio estas
serian desarrolladas especificamente para realizar las tareas requeridas pero debido al limitado tiempo
disponible se tomo la decicion de utilizar herramientas ya disponibles que tuviesen caracteristicas
similares a las requeridas en donde su modificacion requeriria menos tiempo que el desarrollo completo
de estas.

Para crear la base del sistema a utilizar se decidio por utilizar uno de los CMS\footnote{"we"} mas 
famosos, Wordpress. Esta herramienta permite crear una web para manejar contenidos la cual es facil
de instalar y de administrar. 

Wordpress posee una fase de instalacion sencilla debido a que los recursos necesarios son pocos. Se 
necesita poseer un hosting donde alojar el dominio y ademas una base de datos del tipo relacional, mysql
o postgres. Una vez inatalado, la administracion total de este es facil de utilizar ya que posee un
dashboard con todas las opciones necesarias para poder utilizar de forma optima esta herramienta. 

Este CMS, ademas de ser de facil uso, posee la opcion de utilizar plugins para aumentar las capacidades y
funciones del mismo. La utilizacion de estas modificaciones es fundamental para entregar todas las
funcionalidades requeridos por la solucion propuesta.

\begin{itemize}

\item Woocommerce: Es un reconocido modulo para Wordpress que ayuda al usuario a crear una tienda de ventas 
online de forma rapida y sin costo. Al ser integrado al sistema este lo modifica para ser facil de configurar
y administrar. En lo que es frontend, se deben utilizar temas especificamente creados para esta herramienta, y 
estos se pueden encontrar de forma gratuita o de pago. Para la solucion utilizamos el tema visual gratuito llamado
\emph{MyStile}. Woocommerce permite administrar de gran forma las caracteristicas del sistema con una coleccion
de plugins que ayudan a aumentar las funcionalidades de este. 

\item WooCube Pro: Modulo, de pago, encargado de la entrega de puntos a los usuarios luego de la 
realizacion de las tareas definidas. A su vez, es el encargado de dar equivalencia a los puntos asi como
dar validez a estos al ser usados como creditos de la tienda en alguna compra. 

\item WPAchievement: Herramienta de pago que entrega los achievements o medallas a los usuarios
luego de realizada la tarea asociada a alguno de estos.

\item Refer a Friend: Plugin responsable de la administracion de las invitaciones de los clientes a
potenciales usuarios. A su vez se encarga de entregar los beneficios cuando se realize la tarea
definida, en el caso de estudio al comprar por primera vez el cliente invitado.

\end{itemize}
