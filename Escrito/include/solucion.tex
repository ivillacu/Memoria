La presente sección consiste en la presentación de la solución escogida
para mejorar la problemática de las pequeñas empresas para competir en lo
referente a comercio electrónico con sus pares mayores.

\section{Metodología de la solución}

La idea para dar solución a la problemática de las micro empresas,
que consiste en atraer y retener clientes, para así poder competir
con empresas de mayor tamaño, tiene como base el uso de una herramienta
OpenSource, para así crear una tienda virtual atractiva a los usuarios
y que pueda estar a la altura de grandes sitios web.

Las siguientes características son necesarias para poder implementar un sistema
web enfocado a una pequeña empresa, que no posee los recursos necesarios
para un gran despliegue web:

\begin{itemize}
    \item {\bf Bajo costo:}
        Debido al bajo capital que poseen micro o pequeñas empresas,
        ésta característica es bastante crucial.
        La poca viabilidad de utilizar soluciones completas ofrecidas
        por empresas externas, nos lleva a poder utilizar una o más herramientas
        que en su conjunto emulen a un sistema completo que necesita ser
        implementado.

    \item {\bf Facilidad de configuración:}
        Existen diversas herramientas que ayudan a un negocio emergente a crear
        una web, pero pocas estan pensadas para ser configuradas por un usuario
        sin tantos contenidos tecnológicos.
        Esta característica es vital, pues como ya mencionamos,
        se necesita una personas con los conocimientos necesarios tanto
        de instalación como configuración, los cuales pueden ser solucionados
        con la contratación de una persona externa.

    \item {\bf Facilidad de administración:}
        Característica clave dentro de la solución planteada.
        El sistema sera administrado, la mayoria del tiempo, por el dueño de la
        empresa, el cual debe ser capaz de hacer tareas en el sistema
        de manera rápida y frecuentemente, por lo que un grado
        de complejidad en términos de administración jugarán en contran
        a la hora de poder aprovechar el sistema a disposición.
        Adicionalmente, se necesita un sistema de administración rápido,
        para evitar estar mucho tiempo realizando una tarea simple, que
        además de conocimientos del sistema, pueden ser provocados por sistemas
        que no poseen una implementación de funcionalidades ordenadas.

\end{itemize}

Una vez creada la tienda virtual, la empresa entra a competir electrónicamente,
pero en desventaja, con empresas de mayor capital y por ende con tiendas virtuales
de mayor tamaño y que utilizan conceptos para atraer y retener a los clientes.
Para poder competir directamente, es necesario utilizar estos mismo conceptos para
poder alcanzar, de cierta forma, el nivel de visitas y ventas de estos grandes
sitios web.

Para poder disminuir la brecha entre la solución propuesta y una tienda virtual
de gran tamaño utilizaremos el concepto de {\GAM}.
Esta idea ayudará a la tienda a atraer y retener clientes con la utilización
de conceptos del desarrollo de juegos, que ya se han mencionado en éste documento,
de los cuales se han seleccionado los siguientes elementos, como parte
de un esquema clave a implementar:

\begin{itemize}
    \item {\bf Puntuación:}
        Esta herramienta consiste en la entrega de puntos por la realización
        de tareas definidas, como: 

	\begin{itemize}
	\item Inscribirse:  Esta tarea otorga al usuario 500 puntos.
	\item Comentarios en productos: Otorga 100 puntos por cada comentario, 
	con un maximo de 5 comentarios premiables.
	\item Compras realizadas: Se le premia con un $10\%$ de la compra en
	puntos.
	\end{itemize} 

        Estos puntos son equivalentes a crédito en la tienda y pueden ser
        utilizados como parte de pago.
        Estos son utilizados para que los clientes vuelvan a comprar a la tienda,
        utilizando créditos generados por él mismo.

    \item {\bf Achievements (logros):}
        Medallas coleccionables entregadas a los clientes realizando tareas
        definidas, como:

        \begin{itemize}
        \item Inscribirse.
	\item Logearse por primera vez.
        \item Comentar por primera vez.
        \item Primera orden completada.
	\item Compra por una suma mayor a $\$15.000$.
        \end{itemize}

        Esta recompensa es utilizada con el fin de entregar la sensación de éxito,
        e importancia en entre otros compradores y asi mantener una motivación
        de seguir comprando y su estado dentro de la comunidad.

    \item {\bf Referals (Referencias):}
        Invitaciones de usuarios de la tienda que son entregadas a posibles
        clientes para que estos conozcan y compren en la tienda.
        Esta herramienta entrega un beneficio mutuo tanto al que hizo la invitación
        como al invitado. Esta recompensa es un cupon de descuento de 500 puntos.
	En primer lugar el usuario invita a un cliente no suscrito en la tienda enviando
	un cupon valido por 500 puntos. Luego, este al comprar con este cupon reenvia, 
	proceso interno, un cupon de 500 puntos al usuario que lo invito.	
        Esto es utilizado para atraer clientes con gustos similares
        a la tienda, pues el usuario que entregue invitaciones, lo hará a su
        grupo cercano, y así sucesivamente.

\end{itemize}

Al unir la tienda virtual con las ideas propuestas dadas por {\GAM} se crea un
sistema de ventas online capaz de competir con las caracteristicas similares del 
comercio electronico de tienda que poseen más capital y experiencia en este ambito, pero
que estas aun implementan otras herramientas que la ayudan aun mas.

\section{Herramientas}

Para implementar la solucion propuesta se necesitó el uso de variadas herramientas.
En un principio éstas serían desarrolladas específicamente para realizar las
tareas requeridas pero luego de investigar el estado del arte de estas tecnologias
se tomo la decicion de utilizar herramientas ya disponibles que tuviesen caracteristicas 
similares a las requeridas en donde su modificacion requeriria menos tiempo que el desarrollo
completo de estas. Otra caracteristica que ayudo al cambio de paradigma fue que en todas 
las herramientas a utilizar existe una comunidad altamente activa que da soporte y que 
llegaria a ser util al momento de configurar o modificar alguna de estas.

La base del sistema utilizado es Wordpress,
por ser uno de los CMS\footnote{Content management system o Sistema de gestión de contenidos}
 más famosos, y ámpliamente utilizado.
Esta herramienta permite crear una web para manejar contenidos
de una forma fácil, tanto la instalación como la administración de las
funcionalidades básicas del sistema.

Wordpress posee una fase de instalación sencilla debido a que los recursos
necesarios son pocos. Se necesita poseer un \emph{hosting} donde alojar el sitio y dominio,
además de una base de datos relacional, como MySQL o postgres. Una gran ventaja de esta
herramienta es que al necesitar recursos basicos, servidor web y base de datos relacional, 
puede ser instalado de manera facil en la mayoria de los \emph{hostings} a nivel mundial. 
Una vez instalado, la administracion total del sistema es relativamente simple,
ya que posee un \emph{dashboard} con todas las opciones necesarias
tanto para la configuración inicial del sistema, como para la modificación
de algunas características importantes, desde el contenido, hasta el tema
y diseño del sitio.

La características principal de Wordpress, que ha ganado mediante
la comunidad detrás del proyecto, es la variedad de plugins, que
tienen tanto un proceso de instalación fácil, como su administración,
la cual sigue los mismos principios de administración de Wordpress.
Dichos plugins, son el elemento distintivo de cada CMS
y en nuestro caso, han sido los actores principales para aplicar
todos los elementos de {\GAM} en el sistema web.

Los plugins utilizados en nuestro sistema, son los siguientes:

\begin{itemize}

    \item {\bf Woocommerce:}
        Es un reconocido módulo que ayuda al usuario a crear una tienda de
        ventas online de forma rápida y sin costo.
        Al ser integrado al sistema, lo modifica agregando opciones
        extra de configuración y administración.
        El uso de éste plugin genera una restricción en la elección
        del tema (\emph{frontend}) del sistema, pues deben ser compatible
        para que exista consistencia en el sitio.
        Dichos temas existen tanto gratuitos como de pago.
        El tema escogido se llama \emph{MyStile}, el cual no posee costo.
        Woocommerce permite administrar de gran forma las caracteristicas del
        sistema al cual se le pueden agregar una colección de plugins que ayudan a 
	aumentar las funcionalidades de este.

    \item {\bf WooCube Pro:}
	Modulo, de pago, encargado de la entrega de puntos a los usuarios luego de la
	realizacion de las tareas definidas. A su vez, es el encargado de dar equivalencia
	a los puntos asi como dar validez a estos al ser utilizados como creditos 
	en compras realizadas. Este plugin utiliza de base un modulo gratuito, ``Cube points'',
	el cual es la base del diseño de puntaje. Tambien incorpora una herramientas
	para realizar y desplegar una tabla de posciciones con los usuarios que han
	obtenido mayor cantidad de puntaje. 

    \item {\bf WPAchievement:}
	Herramienta de pago que entrega la funcionalidad de otorgar achievements, logros o medallas
	a los usuarios luego de realizada alguna de las tareas definidas, vease \red{Link Cap 4.1}.
	Este modulo es el encargado de la entrega de los logros y tambien del guardado de estos 
	para un futuro despliegue de estos.

    \item {\bf Refer a Friend:}
        Plugin responsable de la administracion de las invitaciones de los clientes a
	potenciales usuarios. A su vez se encarga de entregar los beneficios cuando se 
	realize la tarea definida, en el caso de estudio al comprar por primera vez el cliente invitado.

\end{itemize}
