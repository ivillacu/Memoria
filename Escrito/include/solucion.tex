La presente sección consiste en la presentación de la solución escogida
para mejorar la problemática de las pequeñas empresas para competir en lo
referente a comercio electrónico con sus pares mayores.

\section{Metodología de la solución}

La idea para dar solución a la problemática de las micro empresas,
que consiste en atraer y retener clientes, para así poder competir
con empresas de mayor tamaño, tiene como base el uso de una herramienta
OpenSource, para así crear una tienda virtual atractiva a los usuarios
y que pueda estar a la altura de grandes sitios web.

Las siguientes características son necesarias para poder implementar un sistema
web enfocado a una pequeña empresa, que no posee los recursos necesarios
para un gran despliegue web:

\begin{itemize}
    \item {\bf Bajo costo:}
        Debido al bajo capital que poseen micro o pequeñas empresas,
        ésta característica es bastante crucial.
        La poca viabilidad de utilizar soluciones completas ofrecidas
        por empresas externas, nos lleva a poder utilizar una o más herramientas
        que en su conjunto emulen a un sistema completo que necesita ser
        implementado.

    \item {\bf Facilidad de configuración:}
        Existen diversas herramientas que ayudan a un negocio emergente a crear
        una web, pero pocas estan pensadas para ser configuradas por un usuario
        sin tantos contenidos tecnológicos.
        Esta característica es vital, pues como ya mencionamos,
        se necesita una personas con los conocimientos necesarios tanto
        de instalación como configuración, los cuales pueden ser solucionados
        con la contratación de una persona externa.

    \item {\bf Facilidad de administración:}
        Característica clave dentro de la solución planteada.
        El sistema sera administrado, la mayoria del tiempo, por el dueño de la
        empresa, el cual debe ser capaz de hacer tareas en el sistema
        de manera rápida y frecuentemente, por lo que un grado
        de complejidad en términos de administración jugarán en contran
        a la hora de poder aprovechar el sistema a disposición.
        Adicionalmente, se necesita un sistema de administración rápido,
        para evitar estar mucho tiempo realizando una tarea simple, que
        además de conocimientos del sistema, pueden ser provocados por sistemas
        que no poseen una implementación de funcionalidades ordenadas.

\end{itemize}

Una vez creada la tienda virtual, la empresa entra a competir electrónicamente,
pero en desventaja, con empresas de mayor capital y por ende con tiendas virtuales
de mayor tamaño y que utilizan conceptos para atraer y retener a los clientes.
Para poder competir directamente, es necesario utilizar estos mismo conceptos para
poder alcanzar, de cierta forma, el nivel de visitas y ventas de estos grandes
sitios web.

Para poder disminuir la brecha entre la solución propuesta y una tienda virtual
de gran tamaño utilizaremos el concepto de {\GAM}.
Esta idea ayudará a la tienda a atraer y retener clientes con la utilización
de conceptos del desarrollo de juegos, que ya se han mencionado en éste documento,
de los cuales se han seleccionado los siguientes elementos, como parte
de un esquema clave a implementar:

\begin{itemize}
    \item {\bf Puntuación:}
        Esta herramienta consiste en la entrega de puntos por la realización
        de tareas definidas o compras en la tienda virtual.
        \cmf{Cuales son las tareas definidas?}
        Estos puntos son equivalentes a crédito en la tienda y pueden ser
        utilizados como parte de pago.
        Estos son utilizados para que los clientes vuelvan a comprar a la tienda,
        utilizando créditos generados por él mismo.
        \cmf{Ya estamos en la sección solución, así que creo que esto debería
        ser más específico. Cuantos puntos hay por venta, por invitar amigos,
        no sé, por cualquier mecanismo, y de la misma forma, como estos puntos
        se traducen a dinero para comprar en el sitio? o cada producto cuesta
        X pesos, Y puntos}

    \item {\bf Achievements (logros):}
        Medallas coleccionables entregadas a los clientes realizando tareas
        definidas.
        \cmf{Qué tareas definidas?}
        Esta recompensa es utilizada con el fin de entregar la sensación de éxito,
        e importancia en entre otros compradores y asi mantener una motivación
        de seguir comprando y su estado dentro de la comunidad.

    \item {\bf Referals (Referencias):}
        Invitaciones de usuarios de la tienda que son entregadas a posibles
        clientes para que estos conozcan y compren en la tienda.
        Esta herramienta entrega un beneficio mutuo tanto al que hizo la invitación
        como al invitado, una vez que realize una tarea definida.
        \cmf{Qué tarea definida?}
        Esto es utilizado para atraer clientes con gustos similares
        a la tienda, pues el usuario que entregue invitaciones, lo hará a su
        grupo cercano, y así sucesivamente.

\end{itemize}
\cmf{Por qué usas palabras en inglés si el sitio está en inglés?
para dar cercanía con los términos de juegos reales? o para hacer el nexo
con los elementos de {\GAM} ?}

Al unir la tienda virtual con las ideas propuestas dadas por {\GAM} se crea un
sistema de ventas online capaz de competir con tiendas que poseen más capital
y experiencia en este ambito.
\cmf{Realmente se logra? o es el sistema que queda a la misma altura?
pues las grandes tiendas usan muchos más métodos de atracción de clientes.
No está mal decir que el sistema queda similar, pero no sé si se pueda
realizar una competencia directa}

\section{Herramientas}

Para implementar la solucion propuesta se necesitó el uso de variadas herramientas.
En un principio éstas serían desarrolladas específicamente para realizar las
tareas requeridas pero debido al \red{limitado tiempo disponible}
se tomo la decicion
de utilizar herramientas ya disponibles que tuviesen caracteristicas similares a
las requeridas en donde su modificacion requeriria menos tiempo que el desarrollo
completo de estas.
\cmf{Pésimo argumento. No puedes decir que por que tenías poco tiempo no lo hiciste,
pues quedas como flojo. Creo que deberías decir que luego del estado del arte
te diste cuenta que muchas funcionalidades que querías implementar ya existían
en el ambiente y que preferiste utilizarlas por la confianza de tener
una comunidad de desarrollo detrás más grande, que comenzar las cosas desde cero,
adicionalmente, puedes decir si modificaste alguna o le hiciste algún cambio
respecto a código o cosas más chicas, para adaptarlas a tu solución}

La base del sistema utilizado es Wordpress,
por ser uno de los CMS\footnote{``we'' \cmf{que es ``we'' ?}} más famosos,
y ámpliamente utilizado.
Esta herramienta permite crear una web para manejar contenidos
de una forma fácil, tanto la instalación como la administración de las
funcionalidades básicas del sistema.

Wordpress posee una fase de instalación sencilla debido a que los recursos
necesarios son pocos.
Se necesita poseer un \emph{hosting} donde alojar el sitio y dominio,
además de una base de datos relacional, como MySQL o postgres.
\cmf{Acá deberías argumentar que la mayoría de los servicios de hosting
permiten, tanto agregar un dominio como crear una base de datos de forma
fácil, por eso es tan fácil de instalar}
Una vez instalado, la administracion total del sistema es relativamente simple,
ya que posee un \emph{dashboard} con todas las opciones necesarias
tanto para la configuración inicial del sistema, como para la modificación
de algunas características importantes, desde el contenido, hasta el tema
y diseño del sitio.

La características principal de Wordpress, que ha ganado mediante
la comunidad detrás del proyecto, es la variedad de plugins, que
tienen tanto un proceso de instalación fácil, como su administración,
la cual sigue los mismos principios de administración de Wordpress.
Dichos plugins, son el elemento distintivo de cada CMS
y en nuestro caso, han sido los actores principales para aplicar
todos los elementos de {\GAM} en el sistema web.

Los plugins utilizados en nuestro sistema, son los siguientes:

\begin{itemize}

\item Woocommerce: Es un reconocido modulo para Wordpress que ayuda al usuario a crear una tienda de ventas 
online de forma rapida y sin costo. Al ser integrado al sistema este lo modifica para ser facil de configurar
y administrar. En lo que es frontend, se deben utilizar temas especificamente creados para esta herramienta, y 
estos se pueden encontrar de forma gratuita o de pago. Para la solucion utilizamos el tema visual gratuito llamado
\emph{MyStile}. Woocommerce permite administrar de gran forma las caracteristicas del sistema con una coleccion
de plugins que ayudan a aumentar las funcionalidades de este. 

\item WooCube Pro: Modulo, de pago, encargado de la entrega de puntos a los usuarios luego de la 
realizacion de las tareas definidas. A su vez, es el encargado de dar equivalencia a los puntos asi como
dar validez a estos al ser usados como creditos de la tienda en alguna compra. 

\item WPAchievement: Herramienta de pago que entrega los achievements o medallas a los usuarios
luego de realizada la tarea asociada a alguno de estos.

\item Refer a Friend: Plugin responsable de la administracion de las invitaciones de los clientes a
potenciales usuarios. A su vez se encarga de entregar los beneficios cuando se realize la tarea
definida, en el caso de estudio al comprar por primera vez el cliente invitado.
\item {\bf Woocommerce:}
        Es un reconocido módulo que ayuda al usuario a crear una tienda de
        ventas online de forma rápida y sin costo.
        Al ser integrado al sistema, lo modifica agregando opciones
        extra de configuración y administración.
        El uso de éste plugin genera una restricción en la elección
        del tema (\emph{frontend}) del sistema, pues deben ser compatible
        para que exista consistencia en el sitio.
        Dichos temas existen tanto gratuitos como de pago.
        El tema escogido se llama \emph{MyStile}, el cual no posee costo.
        Woocommerce permite administrar de gran forma las caracteristicas del
        sistema con una colección de plugins que ayudan a aumentar las
        funcionalidades de este.
        \cmf{El plugin Woocommerce tiene más plugins? onda Woocommerce-blaBla?
        o te refieres netamente a funcionalidades?}

    \item {\bf WooCube Pro:}
        Modulo encargado de los puntos

    \item {\bf WPAchievement:}
        ...

    \item {\bf Refer a Friend:}
        ..
\end{itemize}
