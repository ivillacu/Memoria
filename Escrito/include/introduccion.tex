La diferencia en el mercado de ventas ha sido siempre descrita por la gran
brecha que existe entre las grandes y pequeñas empresas, sobre todo con el
pasar del tiempo, donde la tecnología ha llegado a ser un actor principal
tanto en la divulgación, como en el manejo interno de cada institución.

Al no poder competir con las grandes empresas, las pequeñas empresas se ven
limitadas a tener un ámbito reducido y netamente local, sin poder establecer
una expansión tan favorable, lo que hace que las ventas se estanquen.

Desde el comienzo de la masificación del uso de Internet, dicha brecha se ha
visto reducida, por distintos factores: (1) La cantidad de clientes \emph{online}
fue creciendo exponencialmente, (2) Grandes y pequeñas empresas podían acceder
al mismo contenido en Internet, por lo que se desarrollaba una competencia
más justa, y (3) Las ventas ya no estaban limitadas a un conjunto de clientes
cercanos a la tienda física.

Actualmente, diversos sistemas son ampliamente utilizados en la creación
de tiendas virtuales, de los que podemos destacar:
\begin{itemize}

\item Magento\footnote{http://www.magento.com}: Plataforma standalone opensource que es utilizada para crear tiendas 
online de mediana a alta escala, comprada por Ebay.
\item Woocommerce\footnote{http://www.woocommerce.com}: Creado como un plugin de wordpress, su facil
configuracion y matencion lo hacen ser utilizado desde pequeñas empresas hasta grandes multinacionales 
como Cosmopolitan\footnote{http://www.cosmopolitan.com}.
\item Prestashop\footnote{http://www.prestashop.com}: Solucion opensource para tiendas online no tan conocida pero
que ha crecido en uso en los ultimos años. Su gran catalogo de traducciones la hace ser utilizada en todo el mundo.

\end{itemize}  

Adicionalmente, también el desarrollo de técnicas asociadas a ampliar y mantener
la clientela ha sido un factor clave en el desarrollo de las empresas,
de las cuales se destaca {\GAM}, tema principal de la presente memoria.

La idea central de  {\GAM} es poder utilizar elementos de \red{juegos ó videojuegos}
para mantener la atención del consumidor, manteniendo una frecuencia de compras,
para así obtener beneficios en el futuro.

En el presente trabajo, combinaremos un sistema de ventas online, concentrándonos
en proveer un sistema de fácil uso y administración, aplicando la técnica {\GAM}
para ayudar a pequeñas empresas a surgir.

El objetivo principal de la memoria es:

\begin{itemize}
    \item Evaluar el impacto de {\GAM} en ventas de videojuegos online.
\end{itemize}

Los objetivos especificos de la memoria son:
\begin{itemize}
    \item Selección de plataforma e-commerce.
    \item Creación de herramientas, utilizando gamification, que apoyen la
          venta online de videojuegos.
    \item Desarrollo de módulo que unifique las herramientas creadas para sistema
          seleccionado.
    \item Proponer métricas para el análisis del uso de gamification.
    \item Análisis de los datos preliminares obtenidos.
\end{itemize}

Esta memoria estara organizada de la siguiente forma:
\red{pendiente}.
