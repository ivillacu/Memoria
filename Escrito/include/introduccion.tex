Desde que se comenzó a utilizar Internet como medio de ventas ha existido una
distancia clara entre PyMes y grandes empresas. Esto ha 
ido disminuyendo debido a que la tecnología ha aumentado y la población
ha sido alfabetizada digitalmente.

Las PyMes, en las cuales se enfoca esta investigación, han debido
saber utilizar estas tecnologías ya que sin ellas están limitadas a un mercado reducido
 y netamente local, sin poder establecer una expansión tan favorable, lo que hace que las ventas se estanquen.

La utilización de Internet y Sistemas de Información han sido responsables de disminuir 
esta brecha entre empresas debido a tres factores: (1) La cantidad de clientes \emph{on-line}
y su crecimiento exponencial, (2) Micro, pequeñas, medianas y grandes empresas con capacidad
de acceder al mismo contenido en Internet, por lo que se desarrolla una competencia
más justa, y (3) Las ventas ya no están limitadas a un conjunto de clientes
cercanos a la tienda física.

Actualmente, diversos sistemas son ampliamente utilizados en la creación
de tiendas virtuales, de los que podemos destacar:

\begin{itemize}
    \item {\bf Magento \cite{Magento} }:
        Plataforma stand-alone OpenSource que es utilizada para crear tiendas \emph{on-line}
        de mediana a alta escala.
        Esta compañía fue comprada por eBay a pocos años de su lanzamiento.

    \item {\bf Wordpress \cite{Wordpress} $+$ Woocommerce \cite{Woocommerce}}:
        Creado como un plugin del reconocido CMS Wordpress, su fácil configuración
        y mantención lo hacen ser utilizado por micro o pequeñas empresas hasta
        grandes multinacionales como Cosmopolitan.

    \item {\bf Prestashop} \cite{Prestashop}:
        Solución OpenSource para tiendas \emph{on-line} no tan conocida en la actualidad
        pero ha crecido en uso en los últimos años.
        Contiene un gran catálogo de traducciones.

\end{itemize}

Adicionalmente, la investigación y el desarrollo de técnicas asociadas a ampliar
y mantener la clientela ha sido un factor clave en la evolución de las empresas,
de las cuales se destaca {\gam}, tema principal de la presente memoria.

La idea central de  {\gam} es poder utilizar elementos del diseño de juegos
para mantener la atención del consumidor.

En el presente trabajo, combinará un sistema de ventas \emph{on-line}, concentrándose
en proveer un sistema de fácil uso y administración, aplicando {\gam}
para ayudar a PyMes a surgir.

El objetivo principal de la memoria es:

\begin{itemize}
    \item Evaluar el impacto de {\gam} en ventas de videojuegos \emph{on-line}.
\end{itemize}

Los objetivos específicos de la memoria son:
\begin{itemize}
    \item Selección de plataforma \emph{e-commerce}.
    \item Creación de herramientas, utilizando {\gam}, que apoyen la
          venta \emph{on-line} de videojuegos.
    \item Desarrollo de módulo que unifique las herramientas creadas para sistema
          seleccionado.
    \item Proponer métricas para el análisis del uso de {\gam}.
    \item Análisis de los datos preliminares obtenidos.
\end{itemize}

El presente documento se encuentra organizado de la siguiente forma.
En el capítulo~\ref{ch:gamification} se expone la historia y definición tras
el concepto y los distintos estereotipos de jugadores que existen.

En el capítulo~\ref{ch:desc} se expone el problema a enfrentar.
También se describen los datos asociados a esta problemática.

A continuación, en el capítulo~\ref{ch:solucion}, se da a conocer la solución
propuesta para enfrentar el problema.La primera parte se expone la metodología
de la solución. En la segunda parte se explican todas las herramientas a utilizar
 y cual es su rol en la solución.

El estudio experimental es abordado en el capítulo~\ref{ch:estudio} donde
se muestran y analizan los resultados obtenidos tanto en la implementación
del sitio web y en la encuesta realizada.

Por último en el capítulo~\ref{ch:conclusiones} se exponen las conclusiones
obtenidas luego de la implementación de la solución y del comportamiento de ésta
con los usuarios.
