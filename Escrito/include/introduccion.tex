La diferencia en el mercado de ventas electrónica ha sido demostrada por la gran
brecha que existe entre las micro, pequeñas o grandes empresas, sobre todo con el
pasar del tiempo, donde la tecnología ha llegado a ser un actor principal
tanto en el marketing, ventas y en el manejo interno de cada institución.

Al no poder competir con las grandes empresas, las micro o pequeñas empresas se ven
limitadas a tener un ámbito reducido y netamente local, sin poder establecer
una expansión tan favorable, lo que hace que las ventas se estanquen.

Desde el comienzo de la masificación del uso de Internet, dicha brecha se ha
visto reducida, por tres factores: (1) La cantidad de clientes \emph{online}
y su crecimiento exponencial, (2) Micro, pequeñas y grandes empresas con capacidad
de acceder al mismo contenido en Internet, por lo que se desarrollaba una competencia
más justa, y (3) Las ventas ya no estaban limitadas a un conjunto de clientes
cercanos a la tienda física.

Actualmente, diversos sistemas son ampliamente utilizados en la creación
de tiendas virtuales, de los que podemos destacar:

\begin{itemize}
    \item {\bf Magento}~\footnote{http://www.magento.com}:
        Plataforma stand-alone OpenSource que es utilizada para crear tiendas online
        de mediana a alta escala.
        Esta compañía fue comprada por eBay pocos años luego de su lanzamiento.

    \item {\bf Wordpress $+$ Woocommerce}~\footnote{http://www.woocommerce.com}:
        Creado como un plugin del reconocido CMS Wordpress, su fácil configuración
        y mantención lo hacen ser utilizado por micro o pequeñas empresas hasta
        grandes multinacionales como Cosmopolitan.

    \item {\bf Prestashop}~\footnote{http://www.prestashop.com}:
        Solución OpenSource para tiendas online no tan conocida en la actualidad
        pero ha crecido en uso en los últimos años.
        Su gran catalogo de traducciones la hace ser utilizada en todo el mundo.

\end{itemize}

Adicionalmente, la investigación y el desarrollo de técnicas asociadas a ampliar
y mantener la clientela ha sido un factor clave en el desarrollo de las empresas,
de las cuales se destaca {\GAM}, tema principal de la presente memoria.

La idea central de  {\GAM} es poder utilizar elementos del diseño de juegos
para mantener la atención del consumidor, manteniendo una frecuencia de compras,
para así obtener beneficios en el futuro.

En el presente trabajo, combinaremos un sistema de ventas online, concentrándonos
en proveer un sistema de fácil uso y administración, aplicando la técnica {\GAM}
para ayudar a micro o pequeñas empresas a surgir.

El objetivo principal de la memoria es:

\begin{itemize}
    \item Evaluar el impacto de {\GAM} en ventas de videojuegos online.
\end{itemize}

Los objetivos específicos de la memoria son:
\begin{itemize}
    \item Selección de plataforma e-commerce.
    \item Creación de herramientas, utilizando {\GAM}, que apoyen la
          venta online de videojuegos.
    \item Desarrollo de módulo que unifique las herramientas creadas para sistema
          seleccionado.
    \item Proponer métricas para el análisis del uso de {\GAM}.
    \item Análisis de los datos preliminares obtenidos.
\end{itemize}

Esta memoria está organizada de la siguiente forma:
\red{pendiente}.

\cmf{Ya estas en posición de saber de que va a ser cada capítulo,
por lo que deberías completar este párrafo}
