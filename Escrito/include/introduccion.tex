En el mercado de ventas siempre ha existido una gran brecha entre grandes y pequeñas 
empresas. A medida que ha avanzado el tiempo la diferencia entre estas ha ido aumentando,
en donde las grandes empresas relegan a las pequeñas a solo vender como negocios de 
barrio y con esto limitan sus clientes, por lo tanto siempre llegan a su tope de sus ventas. 

Con la llegada de internet, el universo de clientes crecio exponencialmente. Con esta nueva
tecnologia, esta brecha entre grandes y pequeñas empresas disminuyo pero debido a la diferencia de capital
, monetario y humano, se ha estancado esta brecha. 

A lo largo del uso de internet han aparecido herramientas que han ayudado a crear y
administrar tiendas virtuales. Al comienzo estas eran complejas y dificiles de configurar
y administrar, con lo cual se debia tener una gran cantidad de capital, monetario y humano,
pero en la actualidad estos sistemas han mejorado bastante con lo cual hoy pueden ser
configurados y mantenidos por poco capital, lo que facilita el uso en empresas pequeñas.

Hoy en dia existen diversos sistemas como magento\footnote{http://www.magento.com}, 
woocommerce\footnote{http://www.woocommerce.com}, prestashop\footnote{http://www.prestashop.com},
entre otros. 

De la misma forma, han aparecidos tecnicas para atraer y retener clientes y que son utilizadas
por las grandes empresas, una de estas es \emph{Gamification}. Esta tecnica utiliza elementos
del desarrollo de juegos para atrapar la atencion del consumidor y darle motivos para volver.

En este trabajo combinaremos un sistema de ventas online de facil uso y adminustracion con
la tecnica de \emph{gamification} para ayudar a las pequeñas empresas a atraer y retener 
clientes.

El objetivo principal de la memoria es:
\begin{itemize}
\item Evaluar el impacto de \emph{gamification} en ventas de videojuegos online.
\end{itemize}

Los objetivos especificos de la memoria son:
-
\begin{itemize}                                                                      
\item Selección de plataforma e-commerce.
\item Creación de herramientas, utilizando gamification, que apoyen la venta online 
de videojuegos.
\item Desarrollo de módulo que unifique las herramientas creadas para sistema 
seleccionado.
\item Proponer métricas para el análisis del uso de gamification.
\item Análisis de los datos preliminares obtenidos.     
\end{itemize}

Esta memoria estara organizada de la siguiente forma:

\begin{itemize}                                                                      
\item *.                                            
\item *.                                                                         
\item *.                                                                        
\end{itemize}


    
