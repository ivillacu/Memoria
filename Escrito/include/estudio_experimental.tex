\section{Sistema Web}
\label{cap_estudio}

El sitio web en el cual se implementó la solución presentada fue creado para la micro empresa de
ventas de vídeo juegos en el capítulo \ref{ch:desc}.

Antes de iniciar la implementación de {\GAM}, se tuvieron que establecer los beneficios a otorgar a los 
clientes, los cuales fueron:

\begin{itemize}

\item Registro en la tienda: 500 puntos.
\item Puntos por venta: 10\% del total de la compra.
\item Comentarios: 100 puntos.
\item Referals: Cupón de 5\% en total venta.

\end{itemize}

La equivalencia entre puntos y dinero es de $1-1$, quiere decir que $1$ punto es equivalente a 
$1$ peso. También existía una restricción a la hora de utilizar los puntos como descuentos, ésta 
era que no se podían utilizar para disminuir más de un 7\%.

La investigación e implementación tuvo dos etapas. La primera de reconocimiento y obtención de datos
 sin la implementación de la solución para crear un línea base del comportamiento de las visitas 
y ventas por este medio.

La segunda parte es luego de implementar la solución
propuesta y realizar una comparación para obtener conclusiones de la utilidad de
{\gam} como una alternativa para mejorar las ventas de una micro o pequeña empresa.

\subsection{Datos pre implementación {\gam}}

Esta primera etapa se dió inicio una vez implementada la base web,
\emph{Wordpress + Woocommerce}, con una base de productos, tanto vídeos juegos como
accesorios.
Tuvo una duración de 30 días, desde el 1 al 30 de noviembre del año 2014,
en los cuales se obtuvieron los siguientes datos importantes:


\begin{itemize}
    \item Cantidad de visitas: 315 (visitas únicas).
    \item Cantidad de lecturas: 445.
    \item Cantidad de usuarios inscritos: 0.
    \item Cantidad de ventas realizadas: 0.
\end{itemize}

Esta información ayuda a crear una línea base para poder comparar los
resultados obtenidos, una vez implementada la idea de {\GAM} en el sitio web.

La nula participación en esta etapa es debido a que también se publicitaban
los productos por otros medios como redes sociales. Al publicitar en otros medios, 
las personas se ven atraídas para realizar las compras de modo presencial y la 
tienda virtual deja de ser efectiva para transacciones de ventas.

\subsection{Datos post implementación {\GAM}}

En esta etapa se implementaron los plugins para transformar el sistema web
convencional de ventas \emph{on-line} a uno \emph{gamificado}.
Estos plugins proveen un sistema de acumulación de puntos,
logros y realizar referencias a amigos.

Luego de obtener la base de comparación, esta nueva etapa tiene una duración
de 30 días, desde el 1 al 31 de diciembre del 2014.

Los datos obtenidos son:

\begin{itemize}
    \item Cantidad de visitas post implementación: 476 (visitas únicas).
    \item Cantidad de lecturas: 641.
    \item Cantidad de usuarios inscritos: 0.
    \item Cantidad de ventas realizadas: 0.
\end{itemize}

\subsection{Comparación de datos}

Una vez obtenidos los datos, éstos entregaron información útil para llegar a
conclusiones que serán presentadas más adelante.

Se puede observar que hubo un aumento en la cantidad de visitas y de lecturas.
Esta comparación ayuda a mostrar que se logró cumplir uno de los objetivos,
atraer posibles clientes.
Este tema es importante debido a que si no se obtiene una mejora en la cantidad
de visitantes, es bastante difícil establecer que la solución presentada ayuda
a las micro o pequeña empresa.

Por otro lado, se puede observar que no se logró una mejora en lo que a nuevos
usuarios registrados y ventas se refiere.
En la primera como en la segunda etapa, no hubo usuarios registrados y tampoco
una transacción, venta, realizada.

Luego de obtener los datos durante dos meses se pudo inferir que la utilización
de {\gam}, no fue efectiva durante este período. Esto no quiere decir que el concepto
de {\gam} sea erróneo, sino que el ambiente en el cual fue implementado no ayudó
al éxito de éste. Factores como la falta de difusión de la tienda online,
 o que los beneficios propuestos no incentivaran a
 los clientes a comprar demostrando falta de compromiso con la utilización del
concepto. Otro de los factores importantes, es el conocimiento del público de la tienda 
para generar un sentimiento de seguridad para que confíen en ésta y en el sistema de pagos, 
esto es un factor que hasta las empresas de mayor tamaño deben lidiar.

Analizando los factores y debido a que la información obtenida al reunir los datos por $2$
 meses no fue concluyente para decidir que {\gam} sea o no una herramienta útil, se
desarrolló una encuesta y ésta es explicada en el capítulo a continuación.

\section{Encuesta}

Para complementar el presente trabajo, se realizó una encuesta, anexo \ref{AnexoI}, con el objetivo de
conocer el conocimiento general de las personas con respecto a la existencia
de las técnicas asociadas a {\gam}.
Por otro lado, se aprovechó la oportunidad para saber cuáles son las preferencias
que tiene la gente a la hora de participar en sistemas que utilizan {\gam},
siendo los más conocidos la acumulación de puntos en grandes empresas de retail,
para conseguir beneficios y recompensas.

Finalmente, podremos apreciar si la gente está dispuesta al uso de sistemas,
o ambientes de la vida cotidiana que utilicen características de {\gam},
ya sean aplicadas a la educación, como al área laboral, lo cual demostrará
 si es que existen ciertas restricciones a la hora de aplicar
el principio expuesto en este trabajo, para el común de las personas
encuestadas.

\subsection{Metodología}

Para la validación de la encuesta se utilizó un intervalo de confianza de un 95\%,
se espera que este porcentaje de la población entregue respuestas verdaderas,
con un error muestral de 7.5\%, se estima que la variación de respuestas entre
muestras de población será de cerca de este porcentaje además debido a la
aleatoriedad de la población.
Dicho error muestral es más alto de lo normal, ya que la forma en la cual la
encuesta fue difundida, nos entregará un universo más aleatorio.
Se utilizaron las redes sociales, las cuales se conforman de un universo de personas
con una gran homogeneidad.

Considerando nuestro intervalo de confianza y error muestral,
se necesitan al menos 165 personas encuestadas para obtener una encuesta válida
y que la muestra de población obtenida se una muestra representativa de la
población total.

La encuesta fue difundida vía redes sociales, y fue realizada utilizando la plataforma Google 
Form para un fácil análisis y almacenamiento de todas las respuestas.

La encuesta fue respondida por 180 personas, y los resultados obtenidos son
descritos y analizados a continuación.

\subsection{Datos e información}

De un universo de 180 personas encuestadas, el 71\% son hombres y el 29\%
restante son mujeres.
Los rangos de edades de los encuestados van desde los 18 a 32 años,
estando el $99\%$ de las personas dentro de este rango.

Analizando la pregunta número $3$, ¿Está al tanto de lo que es {\gam}/Ludificación?, se obtiene que  un 57\% 
de los encuestados ya tiene algún conocimiento sobre {\gam}, mostrado en el cuadro \ref{tab:Pregmulti}.
Esta información es relevante al momento de implementar una solución gamificada.
Teniendo en cuenta que un gran porcentaje de la población de la muestra es joven,
se puede inferir que es la población con una capacidad mayor de aceptar el
concepto de {\gam} al ser implementado como solución, en este caso en el
contexto de ventas \emph{on-line}.

La siguiente pregunta,
¿Usted acumula puntos de grandes empresas?, fue creada con el motivo de reforzar
al usuario una forma de {\gam} con la cual se relaciona frecuentemente y así dar
a entender de mejor forma el concepto.
En ésta se obtiene un resultado similar al de la pregunta anterior,
donde un 56\% contesta positivamente, como se muestra en el cuadro \ref{tab:Pregmulti} .
Con esta respuesta se ratifica que existe algún conocimiento sobre {\gam} por parte
de más de la mitad de los encuestados, además se demuestra que han interactuado con
ella de alguna forma.

Las preguntas 5 y 6 son exclusivas para las personas que respondieron positivamente
la pregunta anterior (número 4).
Estas preguntas no son obligatorias por lo cual la muestra poblacional
es menor a la obtenida en la encuesta, ambas tienen universos diferentes.
En la primera existen $92$ encuestados que constituyen el 100\%, cuadro \ref{tab:Pregmulti}.
Esta pregunta tiene como objetivo conocer si los encuestados están en conocimiento
de los beneficios entregados por las empresas.
Con 69 personas contestando positivamente, equivalente a 75\%, se puede apreciar
que los encuestados tienen conocimientos del fin detrás de la acumulación de puntos,
y es esto lo que crea el motivo de volver y seguir comprando en una tienda
\emph{on-line} o presencial.

La siguiente pregunta tiene un universo de 101 personas, cuadro \ref{tab:Pregmulti},
del cual un 79\% contestó de forma positiva.
Con esto se remarca que los encuestados están en conocimiento de toda la
información necesaria para motivarlos a seguir y conquistar su meta.

La siguiente pregunta,
¿Qué beneficios prefiere o preferiría obtener?,
tiene como objetivo conocer los  beneficios más atractivos para el usuario.
Esta pregunta es del tipo \emph{Likert} \cite{likert} consta de una escala de preferencia de
1 a 5, siendo 1 la preferencia menos interesante y 5 la más interesante.
El beneficio con mas preferencias $5$, o más interesante para los encuestados,
es la obtención de descuentos en dinero con un 80\%, cuadro \ref{tab:Preg7}.
El siguiente beneficio con más interesados es la adquisición de descuentos para
próximas compras con un 32\%, cuadro \ref{tab:Preg7}, de los encuestados.
Una información importante obtenida es que una de las herramientas más utilizadas,
obtención de puntos, es resentida por las personas debido a que existe un mayor
rechazo, 23\%, que aceptación, 17\%, pero la diferencia no es sustantiva para
descartarla como método a utilizar.
El beneficio más rechazado, mayor cantidad de 1, fue la obtención de un
reconocimiento por tabla de posiciones con un 81\%.

\begin{table}[h]
\centering
\footnotesize
\begin{tabular}{|l|c|c|c|c|c|c|}
\hline
 & \multicolumn{6}{c|}{{\bf Preferencias(\%)}} \\
\hline
{\bf Opciones} & {\bf 1} & {\bf 2} & {\bf 3} & {\bf 4} & {\bf 5} & {\bf Total}\\
\hline
Precios Bajos & 1\% & 1\% & 5\% & 13\% & 80\% & 180\\
\hline
Acumulación de puntos & 23\% & 22\% & 21\% & 17\% & 17\% & 180\\
\hline
Descuentos en compras & 7\% & 14\% & 18\% & 29\% & 32\% & 180\\
\hline
Obtención de productos exclusivos & 24\% & 17\% & 25\% & 16\% & 18\% & 180\\
\hline
Reconocimiento publico & 82\% & 7\% & 6\% & 2\% & 3\% & 180\\
\hline
\end{tabular}
\caption{Tabla de respuestas para pregunta de la encuesta: {\bf ¿Qué beneficios prefiere o preferiría obtener?}}
\label{tab:Preg7}
\end{table}

La pregunta 8, cuadro \ref{tab:Preg8},
¿Cuál es su preferencia a la hora de realizar compras?,
muestra las preferencias  de compra, presencial u \emph{on-line}.
La preferencia más seleccionada, por 134 personas, fue la opción de compra
presencial.
Esto indica que en Chile no se acostumbra a comprar vía Internet y esto representa
una oportunidad de negocio debido al incremento de alfabetización digital y
con esto el aumento del acceso a Internet.

\begin{table}[h]
\centering
\footnotesize
\begin{tabular}{|p{6cm}|c|c|c|}
\hline
{\bf Pregunta} & {\bf Presencial (persona a persona)(\%)} & {\bf On-line (vía internet)(\%)} & {\bf Total}\\
\hline
¿Cuál es su preferencia a la hora de realizar compras?& 26\% & 74\% & 180 \\
\hline
\end{tabular}
\caption{Tabla de respuesta para pregunta de la encuesta: {\bf ¿Cuál es su preferencia a la hora de realizar compras?}}
\label{tab:Preg8}
\end{table}


Una vez respondida la pregunta anterior se le expone al usuario, en la pregunta 9,
si preferiría utilizar un sitio de ventas \emph{on-line} convencional, sólo es utilizado
para la transacción entre cliente y empresa, o un sitio el cual ofrezca
herramientas de {\gam}.
Con un 69\% seleccionan una tienda con {\gam} como la alternativa preferida.
Con este resultado se remarca la utilización de este concepto con el fin de atraer
a nuevos usuarios.

\begin{table}[h]
\centering
\footnotesize
\begin{tabular}{|p{6cm}|c|c|c|}
\hline
Pregunta & Tienda on-line con gamificación(\%) & Tienda on-line convencional(\%) & Total\\
\hline
¿Prefiere un sitio de ventas on-line con ``\emph{gamification}'' o un sitio convencional?& 69\% & 31\% & 180 \\
\hline
\end{tabular}
\caption{Tabla de respuesta para pregunta de la encuesta: {\bf ¿Prefiere un sitio de ventas on-line con 
``\emph{gamification}'' o un sitio convencional?}}
\label{tab:Preg9}
\end{table}


La pregunta 10, cuadro \ref{tab:Pregmulti}, intenta ratificar una de las mayores ideas tras la utilización
de {\gam}, la motivación.
Este sentimiento se crea en el usuario con el fin de atraer y retener clientes.
Se obtuvo que 122 personas han creado este sentimiento que los impulsa a volver
a utilizar o comprar en la tienda implementada con {\gam}.
Esta información es bastante importante al momento de implementar una solución
gamificada y aún más cuando se trata de comercio electrónico.

\begin{table}[h]
\centering
\footnotesize
\begin{tabular}{| p{6cm} | c | c | c |}
\hline
                          Pregunta
                        & Si(\%)
                        & No(\%)
                        & Total \\ \hline
¿Está al tanto de lo que es {\gam} - ludificación?&57\%&43\%&180 \\ \hline
¿Usted acumula puntos de grandes empresas?&56\%&44\%&180 \\ \hline
¿Está al tanto de los beneficios que dicha empresa le ofrece?&75\%&25\%&92 \\ \hline
¿Ha utilizado éstos beneficios en algún momento?&79\%&21\%&100 \\ \hline
¿Cree usted que el uso de {\gam} lo motiva a volver a la tienda?&68\%&32\%&180 \\ \hline
\end{tabular}
\caption{Tabla de respuestas para preguntas de encuesta numeros 3, 4, 5, 6, 10 }
\label{tab:Pregmulti}
\end{table}


Por último,
se intenta obtener una idea de qué contextos podrían ser interesantes para
implementar una solución gamificada.
Con esto en mente, la más votada fue el contexto social con 109 preferencias,
seguida por educación con 100 encuestados.
Con esta información se pueden inferir ideas para desarrollos futuros en esta área.

Luego de realizar la implementación {\gam} sobre una tienda \emph{on-line} y con los datos
obtenidos mediante la encuesta realizada, se cuenta con bastante información para
analizar y obtener conclusiones que ayudarán a futuras empresas que quieran
utilizar este concepto. También se obtiene información para guiar
trabajos futuros en otros contextos.

