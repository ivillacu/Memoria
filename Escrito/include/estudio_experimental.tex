\section{Sistema Web}

\red{sistema web}


\section{Encuesta}

Como apoyo, se realizo una encuesta con el objetivo de conocer si las personas tenian conocimiento
de la existencia de estas tecnicas de \emph{gamification}. Otra parte de esta buscaba ver los
gustos en lo que beneficios o recompensas se refiere. Por ultimo, conocer si una vez conocido o recordado
el cocepto central de la investigacion, preferirian una tienda convencional sin \emph{gamification} o 
una tienda con el concepto implementado.

\subsection{Metodologia}

Para esta encuesta se utilizo un intervalo de confienza de un $95%$ con un error muestral de $7.5%$. Se
utiliza un error muestral mas alto de lo normal ya que la forma de difución de esta fueron las 
redes sociales y es mas aleatorio la seleccion de estas.

 Con estos datos se obtiene que se necesitan al menos 150 personas encuestadas para que sea valida y
que los resultados extraidos sean confiables. 

La encuesta fue realizada via internet utilizando google form y fue respondida por 161 personas. A 
continuación se mostrara los datos obtenidos y se realizaran algunas conjeturas al respecto.

\subsection{Datos e información}

De las 161 personas encuestadas 114 de ellas son hombres y 47 son mujeres dentro de las edades 
de 18 a 32 años. 

De los datos obtenidos se puede inferir que \emph{gamification} es un concepto conocido por una gran parte
de los jovenes en chile y que ademas un gran porcentaje utiliza la acumulación de puntos como una 
herramienta que demuestra \emph{gamification}.
