\section{Sistema Web}



\section{Encuesta}

Para complementar el presente trabajo, se realizó una encuesta con el objetivo de
conocer el conocimiento general de las personas con respecto a la existencia
de las técnicas asociadas a {\GAM}.

Por otro lado, se aprovechó la oportunidad para saber cuales son las preferencias
que tiene la gente a la hora de participar en sistemas que utilizan {\GAM},
siendo los más conocidos la acumulación de puntos en grandes empresas de retail,
para conseguir beneficios y recompensas.

Finalmente, podremos apreciar si la gente está dispuesta al uso de sistemas,
o ambientes de la vida cotidiana que utilicen características de {\GAM},
ya sean aplicadas a la educación, como al área laboral, lo cual nos podrá
demostrar si es que existen ciertas restricciones a la hora de aplicar
el principio expuesto en este trabajo, para el común de las personas
encuestadas.

\subsection{Metodología}

En la encuesta se utilizó un intervalo de confianza de un 95\% con un error
muestral de 7.5\%.
Dicho error muestral es más alto de lo normal, ya que la forma en la cual la encuesta
fue difundida, nos entregará un universo más aleatorio, ya que se hizo uso de redes
sociales, en las cuales no estamos reduciendo nuestro universo de personas
a un estereotipo determinado.


Con estos datos se obtiene que se necesitan al menos 150 personas encuestadas para que sea valida y
que los resultados extraidos sean confiables. 

Considerando nuestro intervalo de confianza y error muestral,
se necesitan al menos 150 personas encuestadas para demostrar
una validez del estudio y que el grado de confiabilidad de esta
sea confiable.

Como se mencionó anteriormente, la encuesta fue transmitida via redes sociales,
y realizada utilizando la plataforma Google Form para un fácil analisis y
almacenamiento de todas las respuestas.

La encuesta fue respondida por 161 personas,
y los resultados obtenidos son descritos y analizados a continuación.

\subsection{Datos e información}

De un universo de 180 personas encuestadas, el $71\%$ son hombres y el $29\%$ restante son mujeres. 
Los rangos de edades de los encuestados van desde los 18 a 32 años, estando el $99\%$ de las personas
dentro de este rango.

Dentro de la pregunta numero 3, se obtiene que con un $54\%$ ya tiene conocimientoas de la existencia
y utilizacion de \emph{gamification}, siendo el rsto de personas ignorantes sobre el tema. 
\cmf{Por cada línea que haga referencia a los resultados de una pregunta,
deberías incluir una gráfica, como un histograma, de los resultados}

De las 161 personas encuestadas 114 de ellas son hombres y 47 son mujeres dentro de las edades
de 18 a 32 años.


De los datos obtenidos se puede inferir que \emph{gamification} es un concepto conocido por una gran parte
de los jovenes en chile y que ademas un gran porcentaje utiliza la acumulación de puntos como una
herramienta que demuestra \emph{gamification}.
