\section{Sistema Web}

El sitio web en el cual se implemento la solucion presentada fue creado para una micro empresa de
ventas de video juegos, llamada Kurgan, ubicada en avenida Valparaíso 595 Local 9, Viña del Mar.
Su existencia es desde el año 2010 y su especialidad es la venta de consolas, video juegos y
accesorios. Tambien se dedica a la venta de juegos de mesa y figuras coleccionables, en menor medida.

Antes de iniciar la implementacion de {\GAM}, se tuvieron que establecer los beneficios dados a los 
clientes, los cuales eran:

\begin{itemize}

\item Registro en la tienda: 500 puntos.
\item Puntos por venta: 10\% del total de la compra.
\item Comentarios: 100 puntos.
\item Referals: Cupon de 5\% en total venta.

\end{itemize}

La equivalencia entre puntos y dinero es de $1-1$, quiere decir que $1$ punto es equivalente a 
$1$ peso. Tambien existia una restriccion a la hora de utilizar los puntos como descuentos, este 
era que no se podian utilizar para disminuir mas de un 7\%.

La investigacion e implementacion tuvo $2$ etapas. La primera de reconocimiento y obtencion de 
datos sin la implementacion de la solucion para crear un linea base del comportamiento de las visitas
y ventas por este medio. La segunda parte es luego de implementar las ideas propuestas tras la solucion
propuestas y realizar una comparacion para obtener conclusiones de la utilidad de \emph{gamification}
como una solucion para mejorar las ventas de una micro o pequeña empresa.

\subsection{Datos pre implementación \emph{gamification}}
 
Esta primera etapa se dio inicio una vez implementada la base web, wordpress + woocommerce, con
una base de productos, tanto videos juegos como accesorios. Tuvo una duracion de 30 dias, desde el 1
 al 30 de noviembre del año 2014, en los cuales se obtuvieron los siguientes datos importantes:

\begin{itemize}
\item Cantidad de visitas: 315 (visitas unicas).
\item Cantidad de lecturas: 445.
\item Cantidad de usuarios inscritos: 0.
\item Cantidad de ventas realizadas: 0.
\end{itemize}

Esta informacion ayudara a crear una linea base para poder comparar los reultados obtenidos una vez
implementada la idea de \emph{gamification} en el sitio web. 

\subsection{Datos post implementación \emph{gamification}}

En esta etapa se implementaron los pluigins para transformar el sistema web convencional de ventas
online a uno \emph{gamificado}. Estos plugins son los que implementan la acumulacion de puntos, 
achievements y el encargado de realizar las referencias a los amigos.

Luego de obtener la base de comparacion, esta nueva etapa tiene una duración de 30 dias, desde 
el 1 al 31 de diciembre del 2014. Los datos obtenidos son:

\begin{itemize}
\item Cantidad de visitas post implementacion: 476 (visitas unicas).
\item Cantidad de lecturas: 641.
\item Cantidad de usuarios inscritos: 0.
\item Cantidad de ventas realizadas: 0.
\end{itemize}

\subsection{Comparación de datos}

Una vez obtenidos los datos estos entregaron informacion util para llegar a conclusiones que seran 
presentadas mas adelante. 

Se puede observar que hubo un aumento en la cantidad de visitas y de lecturas. Esta comparacion 
ayuda a mostrar que se logro cumplir uno de los objetivos, atraer posibles clientes, de implementar
 \emph{gamification}. Este tema es importante debido a que si no se obtiene una mejora en la cantidad
de visitantes es bastante dificil establecer que la solucion presentada ayuda a las micro o pequeña
empresa.

Por otro lado se puede observar que no se logro una mejora en lo que a nuevos usuarios registrados y 
ventas se refiere. En la primera como en la segunda etapa, no hubo usuarios registrado y tampoco
una transaccion, venta, realizada.
Luego de obtener estos datos durante $2$ meses se podria inferir que la utilizacion de {\GAM} no fue 
efectiva durante este periodo, pero tambien pudo deberse a otros factores como la falta de difusion 
de la tienda o que los beneficios propuestos no incentivaran a los clinetes a comprar, bajos descuentos,
baja equivalencia de puntos o que los clientes suelen comprar por internet en lugares mas reconocidos.

Debido a que la informacion obtenida al reunir los datos por 2 meses no es conluyente para decir que
la utilizacion de {\GAM} no obtuvo los resultados esperados al ser implemenada en una micro empresa se
diseño una encuesta con el objetivo de apoyar la utilizacion de {\GAM} en tiendas online.

A continuacion se desarrolla y explica toda la encuesta realizada. Se analizan los datos obtenidos 
con esta pregunta a pregunta. 

\section{Encuesta}

Para complementar el presente trabajo, se realizó una encuesta con el objetivo de
conocer el conocimiento general de las personas con respecto a la existencia
de las técnicas asociadas a {\GAM}.
Por otro lado, se aprovechó la oportunidad para saber cuales son las preferencias
que tiene la gente a la hora de participar en sistemas que utilizan {\GAM},
siendo los más conocidos la acumulación de puntos en grandes empresas de retail,
para conseguir beneficios y recompensas.

Finalmente, podremos apreciar si la gente está dispuesta al uso de sistemas,
o ambientes de la vida cotidiana que utilicen características de {\GAM},
ya sean aplicadas a la educación, como al área laboral, lo cual demostrara
 si es que existen ciertas restricciones a la hora de aplicar
el principio expuesto en este trabajo, para el común de las personas
encuestadas.

\subsection{Metodología}

Para la validacion de la encuesta se utilizó un intervalo de confianza de un 95\%, se espera que
este porcentaje de la poblacion entregue respuestas verdaderas, con un error
muestral de 7.5\%, se estima que la variacion de respuestas entre muestras poblacionales
 sera de cerca de este porcentaje ademas debido a la aleatoriedad de la poblacion.
Dicho error muestral es más alto de lo normal, ya que la forma en la cual la encuesta fue difundida,
 nos entregará un universo más aleatorio. Se utilizaron las redes sociales las cuales 
se conforman de un universo de personas con una gran homogeneidad.

Considerando nuestro intervalo de confianza y error muestral, se necesitan al menos 165
personas encuestadas para obtener una encuesta valida y que la muestra de poblacion obtenida
se una muestra representativa de la poblacion total.

Como se mencionó anteriormente, la encuesta fue difundida via redes sociales,
y fue realizada utilizando la plataforma Google Form para un fácil analisis y almacenamiento
 de todas las respuestas.

La encuesta fue respondida por 180 personas, y los resultados obtenidos son descritos y
analizados a continuación.

\subsection{Datos e información}

De un universo de 180 personas encuestadas, el $71\%$ son hombres y el $29\%$ restante son mujeres.
Los rangos de edades de los encuestados van desde los 18 a 32 años, estando el $99\%$ de las personas
dentro de este rango.

Analizando la pregunta numero 3, ¿está al tanto de lo que es {\GAM}/Ludificación?, se obtiene que
 un $56\%$ de los encuestados ya tiene algun conocimiento sobre {\GAM}, mostrado en el grafico~\ref{fig:chart5.1}.
Esta informacion es relevante al momento de implementar una solucion gamificada. 
Teniendo en cuenta que un gran porcentaje de la poblacion de muestra es joven se puede infiere 
que es la poblacion con una capacidad mayor de aceptar el concepto de {\GAM} al ser implementado 
como solucion, en este caso en el contexto de ventas on-line.

\begin{figure}[!htb]
  \centering
  \includegraphics[width=0.6\textwidth]{images/Graficos/graf_5_1.png}
  \caption[chart5.1]{Respuesta a pregunta $3$, conocimiento de {\GAM}.}
  \label{fig:chart5.1}
  %\url{http://www.chartgo.com/get.do?id=69b1dc7d4e}
\end{figure}

La siguiente pregunta, ¿usted acumula puntos de grandes empresas? , fue creada con el motívo de 
reforzarle al usuario una forma de {\GAM} con la cual se relaciona frecuentemente y asi darle
a entender de mejor forma el concepto. En esta se obtiene un resultado similar al obtenido en 
la pregunta anterior, donde un $56\%$ contesta positivamente. Con esta respuesta se ratifica que existe
algun conocimiento sobre {\GAM} por parte mas de la mitad de los encuestados, ademas se demuestra que
han interactuado con ella de alguna forma.

\begin{figure}[!htb]
  \centering
  \includegraphics[width=0.6\textwidth]{images/Graficos/graf_5_2.png}
  \caption[chart5.2]{Respuesta a pregunta $4$, acumulacion de puntos.}
  \label{fig:chart5.2}
  %\url{http://www.chartgo.com/get.do?id=69b1dc7d4e}
\end{figure}

Las preguntas $5$ y $6$ son exclusivas para las personas que respondieron positivamente la
pregunta anterior, numero $4$.Estas preguntas no son obligatorias por lo cual la muestra poblacional
es menor a la obtenida en la encuesta, ambas tienen universos diferentes.
En la primera existen $92$ encuestados que contituyen el $100\%$. Esta pregunta tiene como objetivo 
conocer si los encuestados estan en conocimiento de los beneficios entregados por las empresas.  
Con $69$ personas contestando positivamente, equivalente a $75\%$, se puede apreciar que los 
encuestados tienen conocimientos del fin detras de la acumulacion de puntos, y es esto lo que
crea el motivo de volver y seguir comprando en una tienda on-line o presencial.

La siguiente pregunta tiene un universo de $101$ personas, del cual un $78\%$ contesto de forma positiva.
Con esto se remarca que los encuestados estan en conocimiento de toda la informacion necesaria 
para motivarlos a seguir y conquistar su meta.

\begin{figure}[!htb]
  \centering
  \includegraphics[width=0.6\textwidth]{images/Graficos/graf_5_3.png}
  \caption[chart5.3]{Respuesta a pregunta $5$, conocimiento de beneficios.}
  \label{fig:chart5.3}
  %\url{http://www.chartgo.com/get.do?id=69b1dc7d4e}
\end{figure}

\begin{figure}[!htb]
  \centering
  \includegraphics[width=0.6\textwidth]{images/Graficos/graf_5_4.png}
  \caption[chart5.4]{Respuesta a pregunta $6$, utilización de los beneficios.}
  \label{fig:chart5.4}
  %\url{http://www.chartgo.com/get.do?id=69b1dc7d4e}
\end{figure}


La siguiente pregunta, ¿qué beneficios prefiere o preferiría obtener?, tiene como objetivo conocer los
 beneficios mas atractivos para el ususario. Esta pregunta es del tipo \emph{LikeIt}, consta de una 
escala de preferencia del $1$ al $5$, $1$ siendo la preferencia menos interesante y 5 la más interesante.
 El beneficio con mas preferencias $5$, o mas interesante para los encuestados, es la obtencion de
descuentos en dinero con un $79\%$. El siguiente beneficio con mas interesados es la adquisicion 
de descuentos para proximas compras con un $32\%$ de los encuestados. Una informacion importante 
obtenida es que una de las herramientas mas utilizadas, obtencion de puntos, es resentida por las
personas debido a que existe un mayor rechazo, $22\%$, que aceptacion, $18\%$, pero la diferencia 
no es sustantiva para descartarla como metodo a utilizar.
El beneficio mas rechazado, mayor cantidad de $1$, fue la obtencion de un reconocimiento por tabla
 de posiciones con un $81\%$.


\begin{figure}[!htb]
  \centering
  \includegraphics[width=0.6\textwidth]{images/Graficos/graf_5_5.png}
  \caption[chart5.5]{Respuesta a pregunta $7$, cantidad de preferencias a beneficio de precios bajos.}
  \label{fig:chart5.5}
  %\url{http://www.chartgo.com/get.do?id=69b1dc7d4e}
\end{figure}

\begin{figure}[!htb]
  \centering
  \includegraphics[width=0.6\textwidth]{images/Graficos/graf_5_6.png}
  \caption[chart5.6]{Respuesta a pregunta $7$, cantidad de preferencias a beneficio de acumulacion de puntos.}
  \label{fig:chart5.6}
  %\url{http://www.chartgo.com/get.do?id=69b1dc7d4e}
\end{figure}

\begin{figure}[!htb]
  \centering
  \includegraphics[width=0.6\textwidth]{images/Graficos/graf_5_7.png}
  \caption[chart5.7]{Respuesta a pregunta $7$, cantidad de preferencias a beneficio de descuento en 
proximas compras.}
  \label{fig:chart5.7}
  %\url{http://www.chartgo.com/get.do?id=69b1dc7d4e}
\end{figure}

\begin{figure}[!htb]
  \centering
  \includegraphics[width=0.6\textwidth]{images/Graficos/graf_5_8.png}
  \caption[chart5.8]{Respuesta a pregunta $7$, cantidad de preferencias a beneficio de reconocimiento 
publico según tabla de posiciones.}
  \label{fig:chart5.8}
  %\url{http://www.chartgo.com/get.do?id=69b1dc7d4e}
\end{figure}


La pregunta $8$, ¿cuál es su preferencia a la hora de realizar compras?, muestra cual es la preferencia 
a la hora de comprar, presencial u online. La preferencia mas seleccionada, por $134$ personas, 
fue la opcion de compra presencial. Esto indica que en Chile no se acostumbra a comprar via internet 
y esto representa una oportunidad de negocio debido al incremento de alfabetizacion digital y 
con esto el aumento del acceso al internet.

\begin{figure}[!htb]
  \centering
  \includegraphics[width=0.6\textwidth]{images/Graficos/graf_5_9.png}
  \caption[chart5.9]{Respuesta a pregunta $8$, preferencia al momento de comprar.}
  \label{fig:chart5.9}
  %\url{http://www.chartgo.com/get.do?id=69b1dc7d4e}
\end{figure}

Una vez respondida la pregunta anterior se le expone al usuario, en la pregunta $9$, si preferiria
utilizar un sitio de ventas online convencional, solo es utilizado para la trasaccion entre cliente
y empresa, o un sitio el cual le ofresca herramientas de \emph{gamification}. Con un $69\%$ seleccionan
una tienda con {\GAM} como la alternativa preferida. Con este resultado se remarca la
utilizacion de este concepto con el fin de atraer a nuevos usuarios.

\begin{figure}[!htb]
  \centering
  \includegraphics[width=0.6\textwidth]{images/Graficos/graf_5_10.png}
  \caption[chart5.10]{Respuesta a pregunta $9$, preferencia al momento de comprar de forma on-line.}
  \label{fig:chart5.10}
  %\url{http://www.chartgo.com/get.do?id=69b1dc7d4e}
\end{figure}


La pregunta $10$ intenta ratificar una de las mayores ideas tras la utilizacion de {\GAM},
la motivacion. Este sentimiento se crea en el usuario con el fin de atraer y retener a este cliente.
 Se obtuvo que $122$ personas han creado este sentimiento que los impulsa a volver a utilizar o comprar 
en la tienda implementada con {\GAM}. Esta informacion es bastante importante al momento de implementar
una solucion gamificada y aun mas cuando se trata de comercio electronico.

\begin{figure}[!htb]
  \centering
  \includegraphics[width=0.6\textwidth]{images/Graficos/graf_5_11.png}
  \caption[chart5.11]{Respuesta a pregunta $10$, motivacion para volver a comprar en una misma tienda.}
  \label{fig:chart5.11}
  %\url{http://www.chartgo.com/get.do?id=69b1dc7d4e}
\end{figure}

Por ultimo, se intenta obtener una idea de que contextos podrian ser interesantes para implementar una
solucion gamificada. Con esto en mente, la mas votada fue el contexto social con $109$ preferencias, 
seguida por educación con $100$ encuestados. Con esta infromacion se pueden inferir ideas para 
desarrollos futuros en esta area.

\begin{figure}[!htb]
  \centering
  \includegraphics[width=0.6\textwidth]{images/Graficos/graf_5_12.png}
  \caption[chart5.12]{Respuesta a pregunta $11$, contextos a los cuales se le podria implementar {\GAM}.}
  \label{fig:chart5.12}
  %\url{http://www.chartgo.com/get.do?id=69b1dc7d4e}
\end{figure}


Luego de realizar la implementacion {\GAM} sobre una tienda online y con los datos obtenidos mediante
la encuesta realizada se obtuvo bastante informacion para analizar y obtener conclusiones que ayudaran 
a futuras empresas que quieran utilizar este concepto en sus empresas. Tambien se obtiene infromacion
para guiar trabajos futuros en otros contextos.
