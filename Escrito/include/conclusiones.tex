El presente trabajo de memoria aborda la problemática que poseen las pequeñas
empresas al tratar de competir con empresas de mayor tamaño, mayor capital
económico o humano, mediante la utilización de {\GAM} como herramienta para atraer
y retener clientes.

Analizando los los datos obtenidos en el capitulo \ref{ch:estudio} se puede observar que el
implementar un sitio base nuevo, para una micro o pequeña empresa, con todas las
herramientas para la venta de artículos \emph{on-line}, como video juegos, y además,
introducir el concepto de {\GAM} en el sistema, es una tarea bastante compleja.

Para implementar de forma exitosa el sistema completo se necesitan ciertos factores
complementarios, estos fueron descritos en el capitulo \ref{ch:estudio}. Para lograr
una implementacion exitosa se logro determinar, luego de la implementacion de la tienda web, 
que es necesario que no existan restricciones al utilizar los puntos, una limitacion
en esto es perjudicial y disminuye la motivación del cliente para volver a la tienda. 
Tambien es de utilidad que al momento de la implementacion se realize publicidad
de la tienda para aumentar el numero de visitas y asi mostrar las nuevas herramientas
del sitio web. Por ultimo, es recomendable utilizar medios de pago que sean conocidos
por los clientes para crear el sentimiento de seguridad en ellos, este factor es importante
ya que es el paso final en la transaccion entre empresa y cliente. 

Por otra parte, luego de analizar los datos obtenidos en la encuesta, en el capitulo \ref{ch:estudio},
 se puede observar que {\GAM} es un concepto bastante conocido y utilizado por las personas 
entre 18 y 32 años, debido a que son usuarios de las herramientas de {\GAM} entregadas por
diferentes empresas y que a su vez lo hacen con el fin de utilizar los beneficios entregados 
por estas. El más interesante para el cliente es el obtener descuentos en dinero
que pudiesen ser utilizados de forma inmediata o para proximas compras.

La forma de entrega de beneficios más utilizada en el mercado es el intercambio de puntos. 
Al analizar este punto con lo obtenido en la encuesta se puede determinar que no es 
una forma interesante para el cliente y que se deberia investigar otra forma para 
hacer entrega de los beneficios. Una forma es la entrega de beneficios fijos por nivel 
de usuario y que estos sean otorgados mediante el numero de compras o achievements logrados
dentro de la tienda.

Por otra parte, gracias a la información obtenida de la encuesta, se confirma la
idea principal de {\GAM} la cual es utilizar la motivación como herramienta para
que el usuario sienta la necesidad de volver a utilizar el sistema gamificado y
se espera que una solución, a cualquier problemática, implementando {\GAM} se
enfoque sobre esta idea.

\subsection{Trabajos futuros}

Finalmente, esta investigación propone las siguientes ideas como trabajos futuros:


\begin{itemize}}

\item 

    \item Realizar marketing de la tienda antes o durante la implementación.
          Es importante esto para dar un sentimiento de seguridad al cliente al
          momento de comprar.
          \cmf{Los trabajos futuros son relacionados a lo que tu hiciste,
               ya implementaste el sistema, entonces no puedes hacer marketing
               si la solución ya está implementada, quizás a medias, pero ya está.
               Encuentro que éste punto tiene que ser reescrito como el: Analisis
               de marketing para la constante mejora del sistema implementado
               utilizando nuevas técnicas y probándolas con la clientela o quizás
               un conjunto selecto de cliente. algo asi...}

    \item Utilizar otros sistemas base para implementar {\GAM} en ellos.
          En el capitulo 4, tabla~\ref{tab:comp_tools}, se analizaron 2 sistemas
          mas que podrían se útiles según el tamaño de la empresa.
          \cmf{Wordpress no fue suficiente? si planeas utilizar otro tipo de sistema
          base, hay que argumentar que se quiere solucionar una falencia del sistema
          actual}

    \item Investigar una nueva forma de entrega de beneficios debido a que la
          acumulación de puntos se demostró que no era una de las formas mas
          interesante para el cliente.
          \cmf{Que otras técnicas utilizan los sitios gamificados? existe
          algún caso de éxito de un sitio web con alguna técnica determinada}

    \item Utilizar {\GAM} en otros contextos.
          En este trabajo de memoria se utilizó en ventas pero también es posible
          utilizar este concepto en educación, social o trabajo.
          En la encuesta realizada se obtuvieron datos para este punto en
          particular.
          \cmf{Esto es un sueño, no un trabajo futuro}
\end{itemize}

