El presente trabajo de memoria aborda la problemática que poseen las pequeñas
empresas al tratar de competir con empresas de mayor tamaño, mayor capital
económico o humano, mediante la utilización de {\GAM} como herramienta para atraer
y retener clientes.

Analizando los los datos obtenidos en el capitulo \ref{ch:estudio} se puede observar que el
implementar un sitio base nuevo, para una micro o pequeña empresa, con todas las
herramientas para la venta de artículos \emph{on-line}, como vídeo juegos, y además,
introducir el concepto de {\GAM} en el sistema, es una tarea bastante compleja.

Para implementar de forma exitosa el sistema completo se necesitan ciertos factores
complementarios, estos fueron descritos en el capitulo \ref{ch:estudio}. Para lograr
una implementacion exitosa se logro determinar, luego de la implementación de la tienda web, 
que es necesario que no existan restricciones al utilizar los puntos, una limitación
en esto es perjudicial y disminuye la motivación del cliente para volver a la tienda. 
También es de utilidad que al momento de la implementación se realice publicidad
de la tienda para aumentar el numero de visitas y así mostrar las nuevas herramientas
del sitio web. Por ultimo, es recomendable utilizar medios de pago que sean conocidos
por los clientes para crear el sentimiento de seguridad en ellos, este factor es importante
ya que es el paso final en la transacción entre empresa y cliente. 

Por otra parte, luego de analizar los datos obtenidos en la encuesta, en el capitulo \ref{ch:estudio},
 se puede observar que {\GAM} es un concepto bastante conocido y utilizado por las personas 
entre 18 y 32 años, debido a que son usuarios de las herramientas de {\GAM} entregadas por
diferentes empresas y que a su vez lo hacen con el fin de utilizar los beneficios entregados 
por estas. El más interesante para el cliente es el obtener descuentos en dinero
que pudiesen ser utilizados de forma inmediata o para próximas compras.

La forma de entrega de beneficios más utilizada en el mercado es el intercambio de puntos. 
Al analizar este punto con lo obtenido en la encuesta se puede determinar que no es 
una forma interesante para el cliente y que se debería investigar otra forma para 
hacer entrega de los beneficios. Una forma es la entrega de beneficios fijos por nivel 
de usuario y que estos sean otorgados mediante el numero de compras o achievements logrados
dentro de la tienda.

Por otra parte, gracias a la información obtenida de la encuesta, se confirma la
idea principal de {\GAM} la cual es utilizar la motivación como herramienta para
que el usuario sienta la necesidad de volver a utilizar el sistema gamificado y
se espera que una solución, a cualquier problemática, implementando {\GAM} se
enfoque sobre esta idea.

\subsection{Trabajos futuros}

Finalmente, esta investigación propone las siguientes ideas como trabajos futuros:


\begin{itemize}

\item Realizar una adecuada publicidad sobre el sistema nuevo implementado para
	atraer mas clientes a conocer la nueva tienda online. Además analizar
	la población interesada y mejorar el sistema implementando técnicas 
	enfocadas a esta población.

\item El sistema base utilizado en este trabajo de memoria, Wordpress $+$ Woocommerce, contiene
todo lo necesario para ser utilizado por micro o pequeñas empresas la escalabilidad esta limitada
a la compra de plugins. En el capitulo 4, tabla \ref{tab:comp_tools}, se analizaron 2 sistemas más
que podrían ser útiles si se requiere implementar {\GAM} en tiendas de mayor tamaño ya que contienen
mas utilidades incorporadas en el mismo sistema base, en especial \emph{magento}.

\item Investigar y utilizar otra forma de entrega de beneficios, debido a que la 
acumulación de puntos no es bastante interesante para la población. Una nueva forma podría ser la entrega 
de beneficios fijo por nivel de cliente y por la cantidad de achievements obtenidos.

\end{itemize}

