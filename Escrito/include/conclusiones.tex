El presente trabajo de memoria aborda la problematica que poseen las pequeñas empresas al tratar
de competir con empresas de mayor tamaño, mayor capital economico o humano, mediante la utilizacion
de {\GAM} como herramienta para atraer y retener clientes.

Analizando los los datos obtenidos en el capitulo $5$ se puede observar que el implementar un sitio base 
nuevo, para una micro o pequeña empresa, con todas las herramientas para la venta de articulos online, 
como video juegos, y ademas introducir el concepto de {\GAM} a esta es una tarea bastante compleja. 
Para lograr una implementacion existosa del sistema completo se necesita es un marketing adecuado 
de la empresa, antes o durante la implementacion. Esto ayuda para que el cliente sienta seguridad
debido a que es necesario que el usuario ingrese al sistema para utilizar las herramientas de {\GAM}.
Otro resultado de la informacion obtenida en la implemetacion de {\GAM} es que los beneficios otorgados
deben ser libres de uso debido a que si existiese alguna restriccion estos no tendrian el efecto completo
deseado en los clientes, esto es mal visto y sentido como si estos no existieran.     

Luego de analizar los datos obtenidos de la encuesta se puede observar que {\GAM} es un concepto bastante
conocido y utilizado por las personas, entre 18 y 32 años, y que estan conciente de ello debido a que
desean obtener los beneficios entregados. Los mas interesantes para los clientes son descuentos equivalentes
en dinero ya sea instantaneos o a futuro. Una de las formas para entregar estos beneficios es utilizando la 
acumulacion de puntos pero esta no fue uno de los beneficios con mayor interes por lo tanto existe una
necesidad de buscar otra forma, dentro de {\GAM}, para entregar los beneficios.
Finalmente, gracias a la informacion obtenida de la encuesta, se confirma la idea principal de {\GAM} la
cual es utilizar la motivacion como herramienta para que el ususario sienta la necesidad de volver
a utilizar el sistema gamificado y se espera que una solucion, a cualquier problematica, implementando 
{\GAM} se enfoque sobre esta idea.

\subsection{Trabajos futuros}

