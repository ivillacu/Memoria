El presente trabajo de memoria aborda la problemática que poseen las pequeñas
empresas al tratar de competir con empresas de mayor tamaño, mayor capital
económico o humano, mediante la utilización de {\gam} como herramienta para atraer
y retener clientes.

\section{Objetivos cumplidos}



Analizando los los datos obtenidos en el capítulo \ref{ch:estudio} se puede observar que el
implementar un sitio base nuevo, para una micro o pequeña empresa, con todas las
herramientas para la venta de artículos \emph{on-line}, como vídeo juegos, y además,
introducir el concepto de {\gam} en el sistema, es una tarea bastante compleja.

Para implementar de forma exitosa el sistema completo, se necesitan ciertos factores
complementarios. Luego de la implementación de la tienda web, 
se infiere que no existe restricciones al utilizar los puntos, una limitación
en esto es perjudicial y disminuye la motivación del cliente para volver a la tienda. 
También es de utilidad que al momento de la implementación se realice publicidad
de la tienda para aumentar el número de visitas y así mostrar las nuevas herramientas
del sitio web. Por último, es recomendable utilizar medios de pago que sean conocidos
por los clientes para crear el sentimiento de seguridad en ellos, este factor es importante
ya que es el paso final en la transacción entre empresa y cliente. 

Por otra parte, luego de analizar los datos obtenidos en la encuesta, en el capítulo \ref{ch:estudio},
 se puede observar que {\gam} es un concepto bastante conocido y utilizado por las personas 
entre 18 y 32 años, debido a que son usuarios de las herramientas de {\gam} entregadas por
diferentes empresas, y que a su vez lo hacen con el fin de utilizar los beneficios entregados 
por éstas. El más interesante para el cliente es el obtener descuentos en dinero
que pudiesen ser utilizados de forma inmediata o para próximas compras.

La forma de entrega de beneficios más utilizada en el mercado es el intercambio de puntos. 
Al analizar este punto con lo obtenido en la encuesta se puede determinar que no es 
una forma interesante para el cliente y que se debería investigar otra forma para 
hacer entrega de los beneficios. Una forma es la entrega de beneficios fijos por nivel 
de usuario y que estos sean otorgados mediante el número de compras o achievements logrados
dentro de la tienda.

Gracias a la información obtenida de la encuesta, se confirma la
idea principal de {\gam}, la cual es utilizar la motivación como herramienta para
que el usuario sienta la necesidad de volver a utilizar el sistema gamificado y
se espera que una solución, a cualquier problemática, implementando {\gam} se
enfoque sobre esta idea.

\section{{\GAM} en Chile}

Si bien en Chile el concepto de {\gam} es utilizado hace tiempo por empresas de gran tamaño, no se
puede decir lo mismo para aquellas pequeñas empresas. Para que una de estas empresas pueda
llegar a utilizar la idea de {\gam} ésta se debe hacer ascesorar por especialistas, lo cual 
requiere recursos, monetarios y de tiempo, que no pueden destinar a mejorar su interaccion con el usuario. 

Luego de realizar éste trabajo de memoria se puede obsevar que la utilización de {\gam} en Chile tiene 
un gran impacto en la interaccion entre empresa y usuario debido a que estos ultimos reconocen el uso
periodico de las herramientas que entrega {\gam}, como son los puntos, tabla de posciciones y los 
beneficios que esto trae. 
En esta investigación se realizo una encuesta que apoya directamente el uso de {\gam} en las tienda, tanto
online como establecidas, debido a que el usuario chileno necesita del estimulo que este da, beneficios o prdictos, 
al momento de elegir el lugar donde realizara sus compras. Tomando ésta información se ve necesario que las
pequeñas empresas de Chile tomen el concepto de {\gam} como una herramienta para poder competir con sus 
pares que ya utilizan {\gam} y ven los beneficios de este concepto diariamente.

Una vez analizada toda la información obtenida se puede concluir que {\gam} es un concepto que influye 
en el usuario del país y que el sistema propuesto para incorporar {\gam} en la pequeña empresa
es una herramienta importante y que deberia ayudar, en el mediano a largo plazo, a competir con empresas
de mayor tamaño en lo que tiendas virtuales se refiere. 

\section{Herramientas}

Una parte importante del sistema propuesto como solución son las herramientas utilizadas para su creación. 
La primera herramienta requerida fue el sistema base, \emph{Wordpress + Woocommerce}, el cual luego de 
la implementación demostro ser la mejor opción para ser enfocada en la pequeña empresa debido a si 
facil administración y baja curva de aprendizaje para los nuevos usuarios. Las otra opciones para sistema no lograron
destacar en las caracteristicas necesarias para ser consideradas en esta implementación pero deben ser
tomadas en consideración cuando se desee implementar un sistema para empresas de mayor tamaño.

Las demas herramientas utilizadas en la implementación entregan las funcionalidades necesarias para mostrar
las ideas de {\gam} en la tienda virtual. Estas funcionalidades son la acumulación de puntos, referir a un amigo 
y medallas por logros realizados. 

Si bien cada una de estas herramientas fue desarrollada por una empresa diferente se logró una excelente 
integración entre ellas lo cual facilito la implementación al no necesitar un modulo que ayudara en la comunicación
entre estas herramientas.

Finalmente, con las herramientas seleccionadas se logro realizar una implementación exitosa y se espera que para 
el uso en una pequeña empresa no sea necesario añadir nuevas herramientas. 

\section{Cliente}

Para la implementación de la solución propuesta se conto con la ayuda de ``Kurgan Juegos''. Una vez terminada
la recoleccion de datos y con la informacion obtenida de esto, capítulo \ref{cap_estudio}, se le comunico 
al cliente los problemas obtenidos en esta etapa, no logrando un registro o venta, con el objetivo de obtener 
sus pensamientos al respecto.

Si bien no se logro realizar un registro de usuario o venta atravez de la tienda el cliente quedo conforme con
el sitio tomando en concideracion el alza de visitas que presento la web. Tambien explico la baja de motivación, 
del cliente, en la utilización del sistema. Uno de los factores fue el momento economico en el cual estaba la tienda
el cual no ayudo al momento de presentar una gran cantidad de productos en la tienda virtual. Otro factor, fue la 
mala organización del stock de productos en la tienda el cual fue problema al momento de agregar los productos
a la tienda virtual.

Si bien el cliente quedo conforme con el resultado general del sistema no se logro una gran aceptación de esta
debido a los factores antes descritos. 

\subsection{Trabajos futuros}

En ésta investigación se proponen las siguientes ideas como trabajos futuros:


\begin{itemize}

\item Realizar una adecuada publicidad sobre el sistema nuevo implementado para
	atraer más clientes a conocer la nueva tienda online. Además analizar
	la población interesada y mejorar el sistema implementando técnicas 
	enfocadas a esta población.

\item El sistema base utilizado en este trabajo de memoria, Wordpress $+$ Woocommerce, contiene
todo lo necesario para ser utilizado por micro o pequeñas empresas, la escalabilidad está limitada
a la compra de plugins. En el capítulo 4, tabla \ref{tab:comp_tools}, se analizaron dos sistemas más
que podrían ser útiles si se requiere implementar {\GAM} en tiendas de mayor tamaño, ya que contienen
más utilidades incorporadas en el mismo sistema base, en especial \emph{Magento}.

\item Investigar y utilizar otra forma de entrega de beneficios, debido a que la 
acumulación de puntos no es bastante interesante para la población. Una nueva forma podría ser la entrega 
de beneficios fijo por nivel de cliente y por la cantidad de achievements obtenidos.

\end{itemize}


