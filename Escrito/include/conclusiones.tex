El presente trabajo de memoria aborda la problemática que poseen las pequeñas
empresas al tratar de competir con empresas de mayor tamaño, mayor capital
económico o humano, en lo que a ventas se refiere. Esto mediante la utilización de {\gam} 
como herramienta para atraer y retener clientes con el propósito de incrementar las ventas.

\section{Objetivos cumplidos}

Analizando los datos obtenidos en el capítulo \ref{ch:estudio}, se puede observar que el
implementar un sitio base nuevo, para una micro o pequeña empresa, con todas las
herramientas de venta de artículos \emph{on-line}, además introducir el concepto de {\gam} en el sistema,
 es una tarea bastante compleja.

Para implementar de forma exitosa el sistema completo, se necesitan ciertos factores
complementarios como son el interés de la empresa, estabilidad económica de ésta y la necesidad del cliente 
por utilizar el sistema. Es importante, una vez implementado el sistema, que no existan restricciones 
al utilizar los puntos canjeables, una limitación en esto es perjudicial y disminuye la motivación del 
cliente para volver a la tienda.
También es de utilidad, que al momento de la implementación, se realice publicidad
de la tienda para aumentar el número de visitas y así mostrar las nuevas herramientas. 
Por último, es recomendable utilizar medios de pago que sean conocidos
por los clientes con el objetivo de crear un sentimiento de seguridad en ellos, este factor es importante
ya que la transacción es el paso final entre empresa y cliente.

Por otra parte, luego de analizar los datos obtenidos en la encuesta, en el capítulo \ref{ch:estudio},
 se puede observar que {\gam} es un concepto bastante conocido y utilizado por las personas
entre 18 y 32 años, debido a que son usuarios de las herramientas de {\gam} entregadas por
diferentes empresas(puntos, recompensas, etc), y que a su vez lo hacen con el fin de utilizar los beneficios entregados
por éstas. El beneficio más interesante para el cliente es el obtener descuentos en dinero
que pudiesen ser utilizados de forma inmediata o para próximas compras.

En cuanto a la forma de entrega de beneficios más utilizada en el mercado es el intercambio de puntos.
Al analizar este punto con lo obtenido en la encuesta se puede determinar que no es
una opción interesante para el cliente y que se debería investigar otra forma para
hacer entrega de los beneficios. Una alternativa en la entrega de beneficios fijos por nivel
de usuario, tipos de usuarios,  y que estos sean otorgados mediante el número de compras o achievements logrados
dentro de la tienda.

Gracias a la información obtenida de la encuesta, se confirma la
idea principal de {\gam}, la cual es utilizar la motivación como herramienta para
que el usuario sienta la necesidad de volver a utilizar el sistema y
se espera que una solución, a cualquier problemática, implementando {\gam} se
enfoque sobre esta idea.

\section{{\GAM} en Chile}

Si bien en Chile el concepto de {\gam} es utilizado hace tiempo por empresas de gran tamaño, no se
puede decir lo mismo para aquellas pequeñas empresas. Para que una de estas empresas pueda
llegar a utilizar la idea de {\gam},ésta se debe hacer asesorar por especialistas, lo cual
requiere recursos, tales como monetarios y de tiempo, que no pueden destinar a mejorar su interacción con el usuario.

Luego de realizar éste trabajo de memoria se puede observar que la utilización de {\gam} en Chile tiene
un gran impacto en la interacción entre empresa y usuario debido a que estos últimos reconocen el uso
periódico de las herramientas que entrega {\gam}, como se mencionó antes.
En esta investigación se realizo una encuesta que apoya directamente el uso de {\gam} en las tienda, tanto
on-line como establecidas, debido a que el usuario chileno necesita del estimulo que obtiene, beneficios o productos,
al momento de elegir el lugar donde realizara sus compras. Tomando ésta información se ve necesario que las
pequeñas empresas de Chile tomen el concepto de {\gam} como una herramienta para poder competir con sus
pares que ya utilizan {\gam} y ven los beneficios de este concepto diariamente.

Una vez analizada toda la información obtenida se puede concluir que {\gam} es un concepto que influye
en el usuario del país y que el sistema propuesto para incorporar {\gam} en la pequeña empresa
es una herramienta importante y que debería ayudar, en el mediano a largo plazo, a competir con empresas
de mayor tamaño en lo que tiendas virtuales se refiere.

\section{Herramientas}

Una parte importante del sistema propuesto como solución son las herramientas utilizadas para su creación.
La primera herramienta requerida fue el sistema base, \emph{Wordpress + Woocommerce}, el cual luego de
la implementación demostró ser la mejor opción para ser enfocada en la pequeña empresa debido a su
fácil administración y baja curva de aprendizaje para los nuevos usuarios. Las otra opciones para sistema no lograron
destacar en las características necesarias para ser consideradas en esta implementación pero deben ser
tomadas en consideración cuando se desee implementar un sistema para empresas de mayor tamaño.

Las demás herramientas, \emph{WooCube Pro}, \emph{WPAchievement} y \emph{Refer a Friend}, utilizadas en la 
implementación entregan las funcionalidades necesarias para mostrar las ideas de {\gam} en la tienda virtual. Estas funcionalidades son la acumulación de puntos, referir a un amigo o medallas por logros realizados. Otra característica por
lo cual fueron seleccionadas es el soporte con el cual cuentan y que el costo de estas es bajo.

Si bien cada una de estas herramientas fue desarrollada por una empresa diferente, se logró una excelente
integración entre ellas lo cual facilito la implementación al no necesitar un modulo que ayudara en la comunicación
entre estas herramientas.

Finalmente, con las herramientas seleccionadas se logro realizar una implementación exitosa y se espera que para
el uso en una pequeña empresa no sea necesario añadir nuevas herramientas.

\section{Cliente}

Para la implementación de la solución propuesta se contó con la ayuda de la tienda ``Kurgan''. Una vez terminada
la recolección de datos y con la información obtenida de esto, capítulo \ref{cap_estudio}, se le comunico
al cliente los problemas obtenidos en esta etapa, no logrando un registro o venta, con el objetivo de obtener
sus pensamientos al respecto.

Si bien no se logro realizar un registro de usuario o venta a través de la tienda el cliente quedo conforme con
el sitio tomando en consideración el alza de visitas que presento la web. También explico la baja de motivación,
del cliente, en la utilización del sistema. Uno de los factores fue el momento económico en el cual estaba la tienda
el cual no ayudo al momento de presentar una gran cantidad de productos en la tienda virtual. Otro factor, fue la
mala organización del stock de productos en la tienda el cual fue problema al momento de agregar los productos
a la tienda virtual.

Finalmente, el cliente quedo conforme con el resultado general del sistema aun así no logrando una gran aceptación 
de esta debido a los factores antes descritos.

\subsection{Trabajos futuros}

En ésta investigación se proponen las siguientes ideas como trabajos futuros:


\begin{itemize}

\item Realizar un analisis de estrategias de marketing que ayuden, en un inicio, a promocionar
la nueva implementación de {\gam} en una tienda viertual.

\item El sistema base utilizado en este trabajo de memoria,\emph{Wordpress $+$ Woocommerce}, contiene
todo lo necesario para ser utilizado por micro o pequeñas empresas, la escalabilidad está limitada
a la compra de plugins. En el capítulo 4, tabla \ref{tab:comp_tools}, se analizaron dos sistemas más
que podrían ser útiles si se requiere implementar {\gam} en tiendas de mayor tamaño, ya que contienen
más utilidades incorporadas en el mismo sistema base, en especial \emph{Magento}.

\item Investigar y utilizar otra forma de entrega de beneficios, debido a que la
acumulación de puntos no es bastante interesante para la población. Una nueva forma podría ser la entrega
de beneficios fijo por nivel de cliente y por la cantidad de logros obtenidos.

\end{itemize}


