El presente trabajo de memoria aborda la problemática que poseen las pequeñas
empresas al tratar de competir con empresas de mayor tamaño, mayor capital
económico o humano, mediante la utilización de {\GAM} como herramienta para atraer
y retener clientes.

\cmf{agregar referencias a los capítulos, no sólo poner el número}
Analizando los los datos obtenidos en el capitulo 5 se puede observar que el
implementar un sitio base nuevo, para una micro o pequeña empresa, con todas las
herramientas para la venta de artículos \emph{on-line}, como video juegos, y además,
introducir el concepto de {\GAM} en el sistema, es una tarea bastante compleja.

\cmf{Durante?}
Para lograr una implementación exitosa del sistema completo, se necesita un
marketing adecuado de la empresa, antes o \red{durante} la implementación.
\cmf{No entiendo en qué ayuda el marketing para la seguridad del cliente...}
Esto ayuda para que el cliente sienta seguridad, debido a que es necesario que el
usuario ingrese al sistema para utilizar las herramientas de {\GAM}.
\cmf{Qué es un beficio libre de uso? que no vence?}
Otro resultado de la información obtenida en la implementación de {\GAM} es que
los beneficios otorgados deben ser libres de uso debido a que si existiese alguna
restricción estos no tendrían el efecto completo deseado en los clientes,
esto es mal visto y sentido como si estos no existieran.

Luego de analizar los datos obtenidos en la encuesta, se puede observar que {\GAM}
es un concepto bastante conocido y utilizado por las personas entre 18 y 32 años,
y que \red{están conscientes de ello debido a que desean obtener los beneficios entregados.}
\cmf{Esa frase suena super mal y no sé que quisiste decir}

Lo más interesantes para los clientes son descuentos equivalentes en dinero ya sea
instantáneos o a futuro.

Una de las formas para entregar éstos beneficios es utilizando la acumulación de
puntos, pero no fue uno de los beneficios con mayor interés, por lo tanto existe
una necesidad de buscar otra forma dentro de {\GAM}, para entregar los beneficios.
\cmf{Cual otra forma?}

Por otra parte, gracias a la información obtenida de la encuesta, se confirma la
idea principal de {\GAM} la cual es utilizar la motivación como herramienta para
\cmf{Sienta la necesidad, o tenga la preferencia?}
que el usuario sienta la necesidad de volver a utilizar el sistema gamificado y
se espera que una solución, a cualquier problemática, implementando {\GAM} se
enfoque sobre esta idea.

\subsection{Trabajos futuros}

Finalmente, esta investigación propone las siguientes ideas como trabajos futuros:


\begin{itemize}
    \item Realizar marketing de la tienda antes o durante la implementación.
          Es importante esto para dar un sentimiento de seguridad al cliente al
          momento de comprar.
          \cmf{Los trabajos futuros son relacionados a lo que tu hiciste,
               ya implementaste el sistema, entonces no puedes hacer marketing
               si la solución ya está implementada, quizás a medias, pero ya está.
               Encuentro que éste punto tiene que ser reescrito como el: Analisis
               de marketing para la constante mejora del sistema implementado
               utilizando nuevas técnicas y probándolas con la clientela o quizás
               un conjunto selecto de cliente. algo asi...}

    \item Utilizar otros sistemas base para implementar {\GAM} en ellos.
          En el capitulo 4, tabla~\ref{tab:comp_tools}, se analizaron 2 sistemas
          mas que podrían se útiles según el tamaño de la empresa.
          \cmf{Wordpress no fue suficiente? si planeas utilizar otro tipo de sistema
          base, hay que argumentar que se quiere solucionar una falencia del sistema
          actual}

    \item Investigar una nueva forma de entrega de beneficios debido a que la
          acumulación de puntos se demostró que no era una de las formas mas
          interesante para el cliente.
          \cmf{Que otras técnicas utilizan los sitios gamificados? existe
          algún caso de éxito de un sitio web con alguna técnica determinada}

    \item Utilizar {\GAM} en otros contextos.
          En este trabajo de memoria se utilizó en ventas pero también es posible
          utilizar este concepto en educación, social o trabajo.
          En la encuesta realizada se obtuvieron datos para este punto en
          particular.
          \cmf{Esto es un sueño, no un trabajo futuro}
\end{itemize}

