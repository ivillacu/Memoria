\section{Descripciòn del problema}


\red{Hablare con el profesor sobre el uso de la tienda}


Desde los inicios del internet las empresas han tenido las ganas de utilizar esta herramienta para poder 
incrementar el universo de clientes y tambien sus ventas. Un ejemplo de esto es que en el año $1984$ 
la empresa CompuServe creo y lanzo "Electronic Mall", el primer servicio de comercio electronico\cite{Def:1}. 
Años mas tarde$(1992)$, con la llegada del primer navegador web, la empresa \emph{Book Stacks Unlimited}
 crea un sitio de venta de libros con la opcion de pagar mediante tarjeta de credito.

Actualmente, con la irrupcion de las redes sociales, el universo de clientes al que se puede llegar es a escala 
mundial. Empresas grandes, con un gran capital humano y economico, utilizan estas herramientas para aumentar
aun mas su marketing de productos y ademas poseen los recursos para contratar empresas especialistas en este
ambito. Para empresas con un capital menor, Pymes, es de mayor dificultad contar con estas herramientas debido
a que se debe utilizar una gran parte del capital en la contratacion de expertos para que sean bien utilizadas.

La utilizacion del internet como medio de ventas para las pequeñas empresas es vital ya que su universo de clientes
se expande mas alla de lo local. La falta de recursos para la contratacion de expertos en el tema hace que la 
utilizacion de esta herramienta sea poco eficiente debido a la falta de recursos humanos. Con esto estas empresas
no pueden competir de mejor manera con las grandes empresas.

La problematica nace cuando una empresa pequeña desea desea utilizar el comercio electronico y no cuenta 
los los recursos necesarios, monetarios y humanos, para lograr utilizar esta herramienta con exito. Esta situacion
es aproechada por las grandes empresas para atraer y retener a los clientes con un mayor marketing de productos 
disminuyendo las ventas de las empresas pequeñas en donde muchas de estas terminan cerrando despues de un cierto tiempo.
Esto debe ser atacado debido a que una gran parte de los empleos son entregados por micro y pequeñas empresa,
un ejemplo es Chile donde cerca del 70$\%$ de los empleos son entregados por estas empresas, y si estas son
acabadas por las grandes empresas aumentara el desempleo y a esto puede traer una debacle en la economia al haber menos
 produccion en la zona afectada por este problema.

\begin{table}[h]
\centering
\begin{tabular}{lcc}
             & \multicolumn{1}{l}{\# de Trabajadores} & \multicolumn{1}{l}{\% de ocupacìon} \\
Microempresa & 1-4                                    & 40\%                                \\
Pequeña      & 5-49                                   & 37\%                                \\
Mediana      & 50-199                                 & 13\%                                \\
Grande       & + de 199                               & 10\%                               
\end{tabular}
\caption[lala]{lelele}
\end{table}

Hoy en dia las empresas pequeñas tienen acceso a un sin numero de tecnologias que ayudan a la creacion y administracion 
de una tienda virtual. Estas herramientas son cada vez mas faciles de administrar por una persona con conocimientos o 
tambien sus curvas de aprendizaje son menores que en años anteriores. 

La problematica mayor es como incrementar la base de clientes de forma estable para evitar fluctuaciones en la ventas y
asi poder tener una estabilidad economica. Esta es la base del problema que se quiere resolver devido a que los
negocios pequeños tienen una falta de marketing online y no tienen expocicion al publico objetivo al nivel que empresas 
con mayor capital tienen. La ausencia de esto complica a que estos negocios puedan atraer nuevos clientes para poder
aumentar las ventas. Si esto se cumple existe la otra parte del problema que es la retencion de estos para manterner una constancia en las ventas y asi permanecer durante el tiempo.

La solucion a este problema es de una complejidad alta ya que se debe enfocar en los clientes. Estos son diferentes 
tipos de personas y por lo tanto la diversidad de gustos es alta. Si bien la dificultad de atraer a esta cantidad
de distintos tipos de personas es superlativa, uno puede enfocarse en distintos grupos y enfocarse en cada uno de 
estos y complacer sus gustos por separado. \emph{Gamification} cumple con esta forma de enfrentar este problema 
teniendo elementos de gusto para varios tipos de clientes.

Para atacar esta problematica se utilizaran aplicaciones de bajo costo monetario para poder dar solucion a las empresas y
a su vez para enfrentar el problema de atraccion y retencion de clientes se aplicara la tecnica de \emph{gamification} 
que ayudara a la fidelizacion del cliente con la empresa. 

La solucion completa a esta problematica contiene una tienda virtual facil de configurar y admnistrar ademas de traer 
los plugins necesarios para poder utilizar las ideas tras \emph{gamification} en la tienda. Ademas de entregar la 
aplicacion se investigara y realizara un estudio preeliminar para ver el desempeño de \emph{gamification} en tiendas
online de micro o pequeñas empresas. Con este estudio se podra determinar si la utilizacion de esta idea en ventas 
electronicas es exitosa y entregar datos para que otros tomen la idea y la implementen en sus negocios.


