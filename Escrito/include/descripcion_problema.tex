\section{Descripciòn del problema}


\section{Tienda asociada}

\red{Hablare con el profesor sobre el uso de la tienda}

\section{Antecedentes del problema}

Desde los inicios del internet las empresas han tenido las ganas de utilizar esta herramienta para poder
incrementar el universo de clientes y tambien sus ventas. Un ejemplo de esto es que en el año $1984$
la empresa CompuServe creo y lanzo "Electronic Mall", el primer servicio de comercio electronico\cite{Def:1}.
Años mas tarde$(1992)$, con la llegada del primer navegador web, la empresa \emph{Book Stacks Unlimited}
 crea un sitio de venta de libros con la opcion de pagar mediante tarjeta de credito.

Actualmente, con la irrupcion del internet, el universo de clientes al que se puede llegar es a escala
mundial. Empresas grandes, con un gran capital humano y economico, utilizan estas herramientas para aumentar
aun mas el marketing de productos y ademas poseen los recursos para contratar empresas especialistas en este
ambito. Para empresas con un capital menor, micro y pequeña empresa, es de mayor dificultad contar con estas 
herramientas debido a que la utilizacion de estas tecnologias debe ser aplicado por personas con conocimiento
debido a la dificultad de configuracion y mantencion.

Esta diferencia debe ser combatida debido a que una gran parte de los empleos son entregados por micro y pequeñas
 empresa,un ejemplo es Chile donde cerca del 70$\%$ de los empleos son entregados por estas empresas, y si 
estas son destruidas por las grandes empresas causara que aumente el desempleo y esto puede traer una debacle 
en la economia al haber menos produccion en la zona afectada por este problema.

\begin{table}[h]
\centering
\begin{tabular}{lcc}
             & \multicolumn{1}{l}{\# de Trabajadores} & \multicolumn{1}{l}{\% de ocupacìon} \\
Microempresa & 1-4                                    & 40\%                                \\
Pequeña      & 5-49                                   & 37\%                                \\
Mediana      & 50-199                                 & 13\%                                \\
Grande       & + de 199                               & 10\%
\end{tabular}
\caption[lala]{lelele}
\end{table}

\section{Problematica}

La problematica nace cuando una empresa pequeña desea desea utilizar el comercio electronico y no cuenta
los los recursos necesarios, monetarios y humanos, para lograr utilizar esta herramienta con exito. Esta situacion
es aproechada por las grandes empresas para atraer y retener a los clientes con un mayor marketing de productos
disminuyendo las ventas de las empresas pequeñas en donde muchas de estas terminan cerrando despues de un cierto 
tiempo.

Hoy en dia las micro y pequeñas empresas tienen acceso a un sin numero de tecnologias que ayudan a la creacion y
administracion de una tienda virtual. Estas herramientas son cada vez mas faciles de administrar por una persona 
con conocimientos o tambien sus curvas de aprendizaje son menores que en años anteriores.

La problematica mayor es como incrementar la base de clientes de forma estable para evitar fluctuaciones en la
 ventas y asi poder tener una estabilidad economica. Esta es la base del problema que se quiere resolver debido 
a que los negocios pequeños tienen una falta de marketing online y no tienen expocicion al publico objetivo.
 La falta de expocicion  complica a que estos negocios puedan atraer nuevos clientes para poder
aumentar las ventas. Luego, si se logra entregar buen marketing a la micro o pequeña empresa y su expocicion
en el mercado objetivo aumenta existe otra arista de este problema que es la retencion de estos para 
manterner una constancia en las ventas y asi permanecer durante el tiempo.

La solucion a este problema es de una alta complejidad ya que se debe enfocar en los clientes. Estos son diferentes
tipos de personas y por lo tanto la diversidad de gustos es alta. Si bien la dificultad de atraer a esta cantidad
de distintos tipos de personas es bastante alta, se puede enfocar en distintos grupos y enfocarse en cada uno de
estos y complacer sus gustos por separado. \emph{Gamification} cumple con esta forma de enfrentar la problematica
teniendo elementos que atraen varios tipos de clientes.

Para dar solucion a esta problematica se utilizaran aplicaciones de bajo costo monetario. La solucion contiene 
una tienda virtual facil de configurar y admnistrar ademas de traer los plugins necesarios para poder utilizar
 las ideas tras \emph{gamification} en la tienda. Ademas de entregar la aplicacion se investigara y realizara un
 estudio preeliminar para ver el desempeño de \emph{gamification} en tiendas online de micro o pequeñas empresas.
 Con este estudio se podra determinar si la utilizacion de esta idea en ventas electronicas es exitosa y 
entregar datos para que otros tomen la idea y la implementen en sus negocios.


