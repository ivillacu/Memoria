\section{Descripciòn del problema}


\red{Hablare con el profesor sobre el uso de la tienda}


Desde los inicios del internet las empresas han tenido las ganas de utilizar esta herramienta para poder 
incrementar el universo de clientes y tambien sus ventas. Un ejemplo de esto es que en el año $1984$ 
la empresa CompuServe creo y lanzo "Electronic Mall", el primer servicio de comercio electronico\cite{Def:1}. 
Años mas tarde$(1992)$, con la llegada del primer navegador web, la empresa \emph{Book Stacks Unlimited}
 crea un sitio de venta de libros con la opcion de pagar mediante tarjeta de credito.

Actualmente, con la irrupcion de las redes sociales, el universo de clientes al que se puede llegar es a escala 
mundial. Empresas grandes, con un gran capital humano y economico, utilizan estas herramientas para aumentar
aun mas su marketing de productos y ademas poseen los recursos para contratar empresas especialistas en este
ambito. Para empresas con un capital menor, Pymes, es de mayor dificultad contar con estas herramientas debido
a que se debe utilizar una gran parte del capital en la contratacion de expertos para que sean bien utilizadas.

La utilizacion del internet como medio de ventas para las pequeñas empresas es vital ya que su universo de clientes
se expande mas alla de lo local. La falta de recursos para la contratacion de expertos en el tema hace que la 
utilizacion de esta herramienta sea poco eficiente debido a la falta de recursos humanos. Con esto estas empresas
no pueden competir de mejor manera con las grandes empresas.

La problematica nace cuando una empresa pequeña desea desea utilizar el comercio electronico y no cuenta 
los los recursos necesarios, monetarios y humanos, para lograr utilizar esta herramienta con exito. Esta situacion
es aproechada por las grandes empresas para atraer y retener a los clientes con un mayor marketing de productos 
disminuyendo las ventas de las empresas pequeñas en donde muchas de estas terminan cerrando despues de un cierto tiempo.
Esto debe ser atacado debido a que una gran parte de los empleos son entregados por pequeñas o mediana empresa,
un ejemplo es Chile donde cerca del 70$\%$ de los empleos son entregados por estas empresas, y si estas son
acabadas por las grandes empresas aumentara el desempleo y a esto puede traer una debacle en la economia al haber menos
 produccion en la zona afectada por este problema.

Para 




%En los tiempos actuales las empresas grandes cuentan con recursos casi ilimitados para crear marketing 
%y apoyar las ventas incrementandose año tras año. Esta es la gran diferencia con las denominadas Pymes, 
%pequeña y mediana empresa, en donde el capital es bastante limitado y en donde el marketing de sus productos
%es escaso y el universo de clientes es pequeño.

%Hoy en dia esta diferencia ha ido disminuyendo gracias a la utilizacion del internet como una herramienta
% para hacer marketing e incrementar el universo de clientes, una forma de hacerlo es creando una tienda 
%online. Utilizando esta herramienya es posible incrementar las ventas debido a una mayor exposicion y a que el universo de clientes aumenta exponencialmente. 

%El problema que se presenta actualmente es la necesidad de atraer al cliente objetivo y retenerlo 
%con el objetivo de incrementar las ventas de una forma permanente. 
