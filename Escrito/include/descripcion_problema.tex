\section{Descripción del problema}

La problemática nace cuando una empresa pequeña desea desea utilizar el comercio
electrónico y no cuenta con los recursos necesarios, monetarios y humanos,
para lograr utilizar esta herramienta con éxito.
Esta situación es aprovechada por las grandes empresas para atraer y retener a los
clientes con un mayor marketing de productos disminuyendo las ventas de las
empresas pequeñas en donde muchas de estas terminan cerrando después de un cierto
tiempo.

Hoy en día las micro y pequeñas empresas tienen acceso a un sin número de
tecnologías que ayudan a la creación y administración de una tienda virtual.
Estas herramientas son cada vez mas fáciles de administrar por una persona
con conocimientos básicos debido principalmente a que sus curvas de aprendizaje
son menores que en años anteriores.

La problemática mayor es como incrementar la base de clientes de forma estable
para evitar fluctuaciones en la ventas y así poder tener una estabilidad económica.
Esta es la base del problema que se quiere resolver debido a que los negocios
pequeños tienen una falta de marketing online y no tienen exposición al público
objetivo.
La falta de exposición complica a que éstos negocios puedan atraer nuevos clientes
para poder aumentar las ventas.
Luego, si se logra entregar buen marketing a la micro o pequeña empresa y su
exposición en el mercado objetivo aumenta, existe otra arista de este problema
que es la retención de clientes, para así tener una estabilidad en el tema
de ventas, asegurando la permanencia en el tiempo.

La solución a este problema es de una alta complejidad, ya que se debe enfocar en
los clientes.
Estos son diferentes tipos de personas y por lo tanto la diversidad de gustos es
alta.
Si bien la dificultad de atraer a esta cantidad de distintos tipos de personas
es bastante alta, se puede enfocar en distintos grupos y así por cada segmento
de clientes, complacer gustos por separado.
{\GAM} cumple con esta forma de enfrentar la problemática teniendo elementos que
atraen varios tipos de clientes.

Para dar solución a esta problemática, se utilizarán aplicaciones gratuitas y de
bajo costo monetario. La base del sistema, wordpress, el plugin de comercio
electrónico, woocommerce, y el tema visual, mystile, son gratuitos y OpenSource.
Los demas plugins, herramientas de {\GAM}, son de pago pero de bajo costo.

La solución contiene una tienda virtual fácil de configurar y administrar,
además de incluir los plugins necesarios para la utilización de los principios
de {\GAM} en la tienda.
Además de entregar la aplicación, se investigará y realizará un estudio preliminar
para ver el desempeño de {\GAM} en tiendas online de micro o pequeñas empresas.
Con dicho estudio se podrá determinar si la utilización de ésta idea en ventas
electrónicas es exitosa, adecuándose en cierto negocio,
y así entregar la información necesaria para que la estrategia sea implementada
en otros negocios.

\section{Tienda asociada}

En un principio del estudio se tuvo el apoyo de una pequeña tienda de video juegos
llamada ``Kurgan''. Esta contaba con la problematica de tener una tienda virtual
la cual no lograba competir con empresas de mayor dominio en la venta de video juegos
online. Luego de modificar la pagina web y ver el numero de visitas que esta obtenia, sin
herramientas de {\GAM}, se implementaron las mecanicas nuevas.

\section{Antecedentes del problema}

Desde los inicios del internet las empresas han tenido las ganas de utilizar
esta herramienta para poder incrementar el universo de clientes y también sus
ventas.
Un ejemplo de esto es que en el año 1984 la empresa CompuServe creo y lanzó
``Electronic Mall'', el primer servicio de comercio electrónico~\cite{Def:1}.
Años mas tarde, en 1992, con la llegada del primer navegador web, la empresa
\emph{Book Stacks Unlimited} crea un sitio de venta de libros con la opción de
pagar mediante tarjeta de crédito.

Actualmente, con la irrupción del internet, el universo de clientes al que se
puede llegar es a escala mundial.
Empresas grandes, con un gran capital humano y económico, utilizan estas
herramientas para aumentar aún mas el marketing de productos y además poseen los
recursos para contratar empresas especialistas en este ámbito.
Para empresas con un capital menor, micro y pequeña empresa, es de mayor
dificultad contar con estas herramientas debido a que la utilización de estas
tecnologías debe ser aplicado por personas con conocimiento debido a la dificultad
de configuración y mantención.

Esta diferencia debe ser combatida debido a que una gran parte de los empleos
son entregados por micro y pequeñas empresa, un ejemplo es Chile donde cerca del
70\% de los empleos son entregados por estas empresas, y si estas son destruidas
por las grandes empresas causara que aumente el desempleo y esto puede traer una
debacle en la economía al haber menos producción en la zona afectada por este
problema.

\cmf{Falta la referencia en el texto a esta tabla}

\begin{table}[h]
\centering
\begin{tabular}{lrr}
{\bf Empresa}  & {\bf Número de Trabajadores} & {\bf Porcentaje de ocupación} \\
Micro    & 1 a 4                & 40\% \\
Pequeña  & 5 a 49               & 37\% \\
Mediana  & 50 a 199             & 13\% \\
Grande   & más de 199           & 10\% \\
\end{tabular}
\caption[TamañoEmpresa]{Tamaño de empresa segùn cantidad de empleados y
la cantidad de estas en el mercado.}
\end{table}
